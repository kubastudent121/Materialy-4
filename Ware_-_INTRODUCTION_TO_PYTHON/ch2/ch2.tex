\documentclass[../main.tex]{subfiles}
\begin{document}
\setlength{\parindent}{1.5em}
\chapter{\textbf{Lists, Loops, Logic, and 2D Arrays}}
\section{\textbf{Lists}}
\begin{parcolumns}[colwidths={2=0.3\textwidth}]{2}
	\colchunk{
	\begin{flushleft}
	You\textquotesingle ll encounter lists quite a bit in Python. In fact we already used some in the
previous chapter without telling you. Lists are ordered collections of values. To
create a list, put the elements inside of square brackets like this:
	\begin{lstlisting}[language=Python,numbers=none,showstringspaces=false]
	x = [5.6,2.1,3.4,2.9]
	\end{lstlisting}
	You can make a list of any variable type. Here is a list of strings:
	\begin{lstlisting}[language=Python,numbers=none,showstringspaces=false]
	a = ['electron','proton','neutron']
	\end{lstlisting}
	List elements don\textquotesingle t need to be the same data type, so this list is perfectly valid
	\begin{lstlisting}[language=Python,numbers=none,showstringspaces=false]
	a = ['Ben',90,'Chad',75,'Andrew',22]
	\end{lstlisting}
	You can even define a list of lists:
	\begin{lstlisting}[language=Python,numbers=none,showstringspaces=false]
	a = [[4,3,2],[1,2.5,9],[4.2,2.9,10.5],
	[39.4,1.4],[2.7,98,42,16.2]]
	\end{lstlisting}
	When looking at the numerical lists above, you might be tempted to think of
lists as mathematical matrices.Don\textquotesingle t. Lists are not really designed for mathematics. To convince yourself of this, add the statement
	\begin{lstlisting}[language=Python,numbers=none,showstringspaces=false]
	a[1][2] = 'george'
	\end{lstlisting}
	after the list of lists above and then examine a. You can\textquotesingle t calculate on elements
that aren\textquotesingle t numbers! As another example, execute this code
	\begin{lstlisting}[language=Python,numbers=none,showstringspaces=false]
	import numpy as np
	x = np.array([1,2,3]) # x is a NumPy array
	print(x+x) # adding two NumPy arrays adds 
	the values
	y = [1,2,3] # y is a Python list
	print(y+y) # adding lists concatenates the 
	two lists
	\end{lstlisting}
	Study the output from this code and convince yourself that NumPy arrays are the
	right object for doing math, not lists. However, lists are otherwise quite useful
	in Python. \hyperref[Table 2.1]{Table 2.1} gives a list of common list functions. Note that lists are
	objects with methods (e.g. a.sort()), but can also be used as arguments to
	other functions (e.g. \textcolor{violet}{len}(a). Like many things in Python, the reasons for the
	different syntaxes are known only to those who developed Python. For example,
	why does \textcolor{violet}{len}(a) work, but a.\textcolor{violet}{len}() return an error? The organic nature of the
	development of Python syntax results in some inconsistencies in syntax that just
	need to be learned.
\end{flushleft}
	}
\colchunk{
	\begin{wraptable}{l}{5.5cm}
		\caption{.1 A sampling of \textquotesingle\textquotesingle house$-$keeping\textquotesingle\textquotesingle functions for lists.
}\label{Table 2.1}
	\begin{tabular}{  m{1.5cm} m{2cm} }
a[x] & Access element x
in list a\\
a[x:y:z] & Extract a slice of
list a\\
a.append(x) & Append x to list a\\
a.pop() & Remove the last element of list a.\\
len(a) & Find the number
of elements in a\\
a.insert(x,y) & Insert y at location
x in list a\\
a.sort() & Sort list a from
least to greatest.\\
filter(f,x) & Filter list x using
the criteria function f(see lambda
functions).\\
a.reverse() &  Reverse the order
of list a.\\
a.index(x) & Find the index
where element x
resides.\\
a + b & Join list a to list b
to form one list.\\
max(a) & ind the largest element of a\\
min(a) & Find the smallest
element of a\\
sum(a) & Returns the sum of
the elements of a\\
\end{tabular}
\end{wraptable}
}
\end{parcolumns}
\begin{parcolumns}[colwidths={2=0.3\textwidth}]{2}
	\colchunk{
	\begin{flushleft}
	\section{\textbf{The range Function}}
	Python\textquotesingle s range function represents a sequence of integer values that behaves a
	lot like a list. The syntax of this function is like this
	\begin{lstlisting}[language=Python,numbers=none,showstringspaces=false]
	a = range(10,20,2)
	\end{lstlisting}
	This statement represents a sequence of integers that starts at 10, ends at 18,
	and steps in increments of 2. The \textcolor{violet}{range} function does not include the item at
	the endpoint of the range. In Python 2.x, the \textcolor{violet}{range} function returned an actual
	list object, but in Python 3.x (which we are using) it returns an object called an
	iterator. Rather than store a huge list of numbers, an iterator just stores the rule
	for generating the number at any index. So, if you print the range variable
	\begin{lstlisting}[language=Python,numbers=none,showstringspaces=false]
	print(a)
	\end{lstlisting}
	Python will just tell you that a represents a range, but if you execute this code
	\begin{lstlisting}[language=Python,numbers=none,showstringspaces=false]
	a[2]
	\end{lstlisting}
	you can get an actual value.\\
	The \textcolor{violet}{range} function can also be called with 1 or 2 arguments and default
	values will be assigned to the missing ones
	\begin{lstlisting}[language=Python,numbers=none,showstringspaces=false]
	b = range(3,9) # starts at 3, ends at 8, steps of 1 (default)
	c = range(10) # starts at 0 (default), ends at 9, steps of 1 (default)
	\end{lstlisting}
	\section{\textbf{For Loops and Shorthand Assignments}}
	The primary use for the \textcolor{violet}{range} function in writing \textcolor{violet}{for} loops. A \textcolor{violet}{for} loop iterates
	over each element in a list or range. Here is a \textcolor{violet}{for} loop used to sum a list of values\\
	\begin{lstlisting}[language=Python,numbers=none,showstringspaces=false]
	s = 0
	for x in [3,42,1,9.9]:
		s = s + x
	print(s)
	\end{lstlisting}
	This code iterates over the list [3,42,1,9.9], assigning the variable x each value
	in the list, one by one, until it reaches the end of the list. Notice the required colon
	at the end of the \textcolor{violet}{for} statement that marks the beginning of the loop, and also that
	there is no code indicating the end of the \textcolor{violet}{for} loop. Python uses indentation to
	denote the group of code that will be executed during each iteration. To see this,
	indent the \textcolor{violet}{print}(s) statement to the same level as the statement above it and
	execute the code. Note that the \textcolor{violet}{print} function now executes on each iteration of
	the loop instead of just displaying the final value of s.\\
	\indent You can also iterate over the values from the \textcolor{violet}{range} function, like this
	\begin{lstlisting}[language=Python,numbers=none,showstringspaces=false]
	for y in range(5,50,3):
		print(y)
	\end{lstlisting}
	\end{flushleft}
	}
	\colchunk{
	\begin{wraptable}{r}{4.5cm}
	\begin{tabular}{  m{4cm}  }
 Usually you would just use
a sum function, but we are
learning about loops here.
\end{tabular}
\end{wraptable}

	}
	\end{parcolumns}
	\newpage
	\begin{parcolumns}[colwidths={2=0.35\textwidth}]{2}
	\colchunk{
	\begin{flushleft}
	The \textcolor{violet}{range} function only produces integers, so the most common use case is to
use it with a for loop to index a separate array, like this
	\begin{lstlisting}[language=Python,numbers=none,showstringspaces=false]
	x = np.arange(5,10,.1)
	for n in range(len(x)):
		print(x[n])
	\end{lstlisting}
	You can also iterate directly over the values of an array, like this:
	\begin{lstlisting}[language=Python,numbers=none,showstringspaces=false]
	x = np.arange(5,10,.1)
	for y in x:
		print(y)
	\end{lstlisting}
	To practice \textcolor{violet}{for} loops, let\textquotesingle s build a loop to calculate the sum
	$$\sum_{n=1}^{1000}(\dfrac{1}{n^{2}})$$
	\begin{lstlisting}[language=Python,numbers=none,showstringspaces=false]
	s = 0
	for n in range(1,1001):
		s += 1/n**2
	print(s)
	\end{lstlisting}
	The code s += 1/n**2 is equivalent to s = s + 1/n**2, but runs a little faster
	and is a little bit easier to read (once you get used to it). See \hyperref[Table 2.2]{Table 2.2} for more
	shorthand notations. Here\textquotesingle s another example of a loop used to calculate the value
	of p = 20! (20 factorial) using another shorthand notation
	\begin{lstlisting}[language=Python,numbers=none,showstringspaces=false]
	p = 1
	for n in range(1,21):
		p *= n # Multiply p by n
	print(p)
	\end{lstlisting}
	Remember that the range function starts at the first argument (i.e. 1), and goes
	up to, but not including the second argument, so our product will only go up to
	n = 20.	
	\section{\textbf{Logical Statements}}
	Often we only want to run a section of code only when some condition is satisfied.
	This requires the use of logic. The simplest logic is the \textcolor{violet}{if}\ \textcolor{violet}{elif}\ \textcolor{violet}{else} statement,
	which works like this:
	\begin{lstlisting}[language=Python,numbers=none,showstringspaces=false]
	a = 1
	b = 3
	if a > 0:
		c = 1
	else:
	\end{lstlisting}
	\end{flushleft}
	}
	\colchunk{
	\begin{wraptable}{r}{4.5cm}
		\caption{ A list of shorthand variable reassignment notation.
}\label{Table 2.2}
	\begin{tabular}{  m{1.5cm} m{2cm} }
Operation & Shorthand\\
a = a + 1 & a += 1\\
a = a - 2 & a -= 2\\
a = 5*a & a *= 5\\
a = a/c & a /= c\\
a = a \%{} & 10 a \%{}= 10\\
a = a**3 & a **= 3\\
a = a // & 12 a //= 12
\end{tabular}
\end{wraptable}
	}
	\end{parcolumns}
	\newpage
	\begin{parcolumns}[colwidths={2=0.3\textwidth}]{2}
	\colchunk{
	\begin{flushleft}
	\begin{lstlisting}[language=Python,numbers=none,showstringspaces=false]
	c = 0
	if a >= 0 or b >= 0: # If condition 1 met
		c = a + b
	elif a > 0 and b >= 0: # If condition 2 met
		c = a - b
	else: # If neither condition is met.
		c = a * b
	\end{lstlisting}
	Note the locations of colons in these statements, and also note that indentation
	again defines the regions of code that execute as a group. \hyperref[Table 2.3]{Table 2.3} lists the
	standard elements for constructing logical statements.\\
	\indent A word of caution about comparing Python floats is in order here. Because
	of the way floats are represented in a computer, the number that is stored in a
	float is often not exactly the number that you think it is. For instance, execute the
	following, and study the output:
\begin{lstlisting}[language=Python,numbers=none,showstringspaces=false]
	a = 0.1
	b = 3 * a
	c = 0.3
	print(b==c) # Are they the same number?
	print(b) # It sure looks like they are the same.
	print(c) # It sure looks like they are the same.
	print(" {:.45f} ".format(b)) #b--- out to 45 
	decimal places
	print(" {:.45f} ".format(c)) #c--- out to 45 
	decimal places
\end{lstlisting}
	This can cause problems when you are comparing two numbers that you think
	should be equal but actually aren\textquotesingle t equal in the computer. The take home here
	is that comparing two floats to see if they are equal is a bad idea. A better way
	to check to see if two floats are equal (or close enough that we can say they are
	equal) is to check if the absolute value of their difference is very small, like this:
	\begin{lstlisting}[language=Python,numbers=none,showstringspaces=false]
	a = 0.1
	b = 3 * a
	c = 0.3
	print(abs(b - c) < 1e-10)
	\end{lstlisting}
	\section{\textbf{While Loops}}
	A while loop iterates until a certain condition is met. This loop is a good choice
	when you don\textquotesingle t know beforehand exactly how many iterations of the loop will be
	executed but rather what condition you want to be met before finishing. As an
	example, let\textquotesingle s compute the sum
\begin{equation}\label{2.1}
	\sum_{}^{}(\dfrac{1}{n^{2}})
\end{equation}
	by looping until the terms become smaller than 1$\times$ $10^{-10}$.
	\begin{lstlisting}[language=Python,numbers=none,showstringspaces=false]
	term = 1 # Load the first term in the sum
	s = term # Initialize the sum
	\end{lstlisting}
	\end{flushleft}
	}
	\colchunk{
	\begin{wraptable}{l}{4.5cm}
		\caption{ Python\textquotesingle s logic elements.
}\label{Table 2.3}
	\begin{tabular}{  m{1.5cm} m{2cm} }
$==$ & Equal\\
$ >= $ & Greater than or
equal\\
$> $ & Greater than\\
$<$ &  Less than\\
$<=$ & Less than or equal\\
$!=$ & Not equal\\
$and$ &  True if both conditions joined by and
are true\\
$or$ &  True if either of the
conditions joined by
or are true\\
$not$ & True if the following
condition is false.
\end{tabular}
\end{wraptable}
	}
	\end{parcolumns}
	\begin{parcolumns}[colwidths={2=0.3\textwidth}]{2}
	\colchunk{
		\begin{flushleft}
		\begin{lstlisting}[language=Python,numbers=none,showstringspaces=false]
	n = 1 # Set a counter
	while term > 1e-10: # Loop while term is 
	bigger than 1e-10
		n += 1 # Add 1 to n so that it will 
		count: 2,3,4,5
		term = 1./n**2 # Calculate the next 
		term to add
		s += term # Add 1/n^2 to 
		the running total
		\end{lstlisting}
	his loop will continue to execute until term is less than $10^{-10}$. Note that indentation again defines the extent of the code that will execute each 	iteration. Unlike
	the \textcolor{violet}{for} loop, you have to do your own counting in a \textcolor{violet}{while} loop. Be careful about
	what value n starts at and when it is incremented (n+=1). Also notice that term
	must be assigned prior to the start of the loop. If it isn\textquotesingle t the loop\textquotesingle s first logical test
	will fail and the loop won\textquotesingle t execute at all.
	\subsection*{\textbf{The \textcolor{violet}{break} statement}}
	Sometimes \textcolor{violet}{while} loops are awkward to use because you can get stuck in an infinite loop if your check condition is never true. The 		\textcolor{violet}{break} command is designed to help you here. When \textcolor{violet}{break} is executed in a loop the script jumps to just after
	the end at the bottom of the loop. The \textcolor{violet}{break} command also works with \textcolor{violet}{for} loops.
	Here is our sum loop rewritten with break
		\begin{lstlisting}[language=Python,numbers=none,showstringspaces=false]
		term = 1 # Load the first term in the sum
		s = term # Initialize the sum
		n = 1 # Set a counter
		while term > 1e-10: # Loop while term is bigger than 1e-10
		n += 1 # Add 1 to n so that it will count: 2,3,4,5
		term = 1./n**2 # Calculate the next term to add
		s += term # Add 1/n^2 to the running total
		if n > 1000:
			print("This is taking too long. I'm outta here...")
			break
		\end{lstlisting}
\subsection*{\textbf{The continue statement}}
	Another statement that is used with loops is the \textcolor{violet}{continue} statement. When
	\textcolor{violet}{continue} is used, the remainder of the code for the current iteration is skipped
	and the next iteration of the loop begins. If you wanted to do the sum in \hyperref[2.1]{Eq. (2.1)}
	but only include those terms for which n is a multiple of 3, you could do this
	\begin{lstlisting}[language=Python,numbers=none,showstringspaces=false]
	term = 1 # Load the first term in the sum
	s = term # Initialize the sum
	n = 1 # Set a counter
	while term > 1e-10: # Loop while term is bigger than 1e-10
		n += 1 # Add 1 to n so that it will count: 2,3,4,5
		if n % 3 != 0:
			continue # Skip the rest of the code, start the next iteration
		term = 1./n**2 # Calculate the next term to add
		s += term # Add 1/n^2 to the running total
	\end{lstlisting}
		\end{flushleft}				
	}
	\colchunk{
		\begin{flushleft}
	If you are stuck in a loop,
	you can force your program
	to stop by pressing ctrl+c or
	cmd+c	
		\end{flushleft}
	}
\end{parcolumns}
\newpage
\begin{flushright}
\begin{minipage}[t]{12cm}
	
	Now, when the value of n is not a multiple of 3, the last two lines of code in the
	loop will be skipped.\\
\section{\textbf{2D NumPy Arrays}$-$\textbf{Matrices}}
	We introduced 1D NumPy arrays in the last chapter. The syntax for two-dimensional
	arrays (matrices) builds on this foundation. Small matrices can be built by giving
	the array function an argument that is a list of lists, like this
\begin{lstlisting}[language=Python,numbers=none,showstringspaces=false]
	a = np.array([[1,2,3],[4,5,6],[7,8,9]])
\end{lstlisting}
	which could be interpreted as this matrix:
\begin{equation}\label{2.2}
\begin{bmatrix}
    1       & 2 &  3 \\
    4       &5 & 6   \\
    7       & 8 & 9  \\
\end{bmatrix}
\end{equation}
	Once you have a 2D array, single elements can be extracted with this syntax:
\begin{lstlisting}[language=Python,numbers=none,showstringspaces=false]
	b = a[0,2]
\end{lstlisting}
	This code extracts the number 3, the element in the first row and third column in
	the array a. If you wanted to slice out the following 2$×$2 sub$-$matrix:
\begin{equation}\label{2.3}
\begin{bmatrix}
5 &6\\
8 &9\\
\end{bmatrix}
\end{equation}
	you could do it like this:
\begin{lstlisting}[language=Python,numbers=none,showstringspaces=false]
	b[1:3,1:3] # Slice out a sub-array
\end{lstlisting}
	If you want all of the elements in a given dimension, use the : alone with no numbers surrounding it. For example, the following:
\begin{lstlisting}[language=Python,numbers=none,showstringspaces=false]
	b[:,1:3]
\end{lstlisting}
	would extract all of the rows on columns 1 and 2:
\begin{equation}\label{2.4}
\begin{bmatrix}
2&3\\
5&6\\
8&9\\
\end{bmatrix}
\end{equation}
	This kind of slicing can\textquotesingle t be done with lists.
\subsection*{\textbf{Creating Larger Matrices}}
	You can create larger arrays using several functions from NumPy. To create an
	array full of zeros, you use this syntax:
\begin{lstlisting}[language=Python,numbers=none,showstringspaces=false]
A = np.zeros((100,200))
\end{lstlisting}
\end{minipage}
\end{flushright}
\newpage
\begin{parcolumns}[colwidths={2=0.3\textwidth}]{2}
	\colchunk{
	\begin{flushleft}
	This creates an array with 100 rows and 200 columns, completely full of zeros. The
	argument that you provide to the zeros function, a set of numbers surrounded
	by round parentheses, is a data type called a \textit{tuple}. Tuples behave a lot like lists,
	except they are immutable (i.e. after you create a tuple, you can\textquotesingle t change the
	item values). We won\textquotesingle t use tuples a whole lot in computational physics, but you
	should be aware of what they are since you periodically need to create them as
	arguments for functions. You can also ask a NumPy array what size it is using the
	array\textquotesingle s shape property, like this
\begin{lstlisting}[language=Python,numbers=none,showstringspaces=false]
	s = A.shape
\end{lstlisting}
	Notice that the shape property returns a tuple.
	To create an array full of ones, you use this syntax:
\begin{lstlisting}[language=Python,numbers=none,showstringspaces=false]
	B = np.ones((100,200))
\end{lstlisting}
You can also create an identity matrix like this
\begin{lstlisting}[language=Python,numbers=none,showstringspaces=false]
	I = np.identity(100)
\end{lstlisting}
	Since identity matrices must be square, it only requires one number as an argument rather than a tuple.\\
\subsection*{\textbf{Math with Matrices}}
	Once a 2D array is defined, it behaves as a matrix and you can perform all the
	standard linear algebra functions with SciPy\textquotesingle s linear algebra module. The matrix
	multiplication operator is the $\copyright$ symbol. Here are a few examples illustrating
	matrix operations:
\begin{lstlisting}[language=Python,numbers=none,showstringspaces=false]
	import numpy as np
	import scipy.linalg as la
	
	a = np.array([[1,2],[3,4]]) # Create 2 x 2 matrix
	b = np.array([[5,6],[8,9]]) # Create 2 x 2 matrix
	col = np.array([[3],[4]]) # Create 2 x 1 column vector
	
	c = a.T # Transpose the matrix
	e = a.conj().T # Find conjugate transpose of matrix
	f = a @ b # Matrix multiplication
	g = b @ col # Multiply matrix b to column vector
	d = la.inv(a) # Find inverse of matrix
\end{lstlisting}
	The linalg name space within SciPy provides lots of matrix functions. Pretty
	much anything you learned about in your linear algebra class is available there.
\section{\textbf{Solving a Set of Linear Equations}}
	To illustrate the use of matrix computations, let\textquotesingle s solve a set of linear equations.
	Here is an example of a set of two linear equations, with two unknowns:
\begin{equation}\label{2.5}
3x + y = 9~~~~x +2y = 8
\end{equation}
	\end{flushleft}
	}
	\colchunk{
	\begin{flushleft}
	 NumPy also has a linear algebra module. The SciPy
	version can do all that the
	NumPy version can, plus it
	has more advanced functions. In addition, the SciPy
	linear algebra package
	has some optimizations
	to make it run faster. We
	recommend you always
	use the SciPy linear algebra
	package
	\end{flushleft}
	}
	\end{parcolumns}
	\newpage
	\begin{flushright}
	\begin{minipage}[t]{12cm}
	This problem can be represented in matrix form like this:
	\begin{equation}\label{2.6}
	\textbf{Ax~ = ~b}
	\end{equation}
	where
	\begin{equation}\label{2.7}
	\textbf{A}
		=
	\begin{bmatrix}
    	3 &  1 \\
    	1 & 2
	\end{bmatrix}
	\end{equation}
	\begin{equation}\label{2.8}
	\textbf{b}
	=
	\begin{bmatrix}
    	9   \\
    	8  
	\end{bmatrix}
\end{equation}
	and
	\begin{equation}\label{2.9}
\textbf{A}
=
\begin{bmatrix}
    x  \\
    y 
\end{bmatrix}
\end{equation}
	SciPy has a function called solve that will solve this problem like this
\begin{lstlisting}[language=Python,numbers=none,showstringspaces=false]
	import numpy as np
	import scipy.linalg as la
	
	a = np.array([[3,1],[1,2]])
	b = np.array([[9],[8]])
	x = la.solve(a,b)
\end{lstlisting}
\section{\textbf{ NumPy Matrix}}
	NumPy provides another object called a matrix that has been widely used for
	linear algebra to represent matrices. The matrix object is a subclass of the NumPy
	array and provides a little cleaner syntax for manipulating matrices than the array
	syntax. However, as of Python 3.5, the developers of NumPy have indicated that
	they plan to deprecate the usage of the matrix type, so we haven\textquotesingle t taught it here.
	But you may still see it in existing code bases. It is essentially a wrapper around
	the NumPy array object to make some linear algebra syntax cleaner. All of the
	functionality of the matrix object is available with NumPy arrays with some small
	syntax changes.
	\end{minipage}
	\end{flushright}
	\clearpage
\end{document}