\documentclass[../main.tex]{subfiles}
\begin{document}
\chapter{Coupled Nonlinear Oscillators}
\begin{minipage}[t]{12cm}
\setlength{\parindent}{1.5em}
	When two or more oscillators are hooked together, we say that they are coupled\textsuperscript{1} Our final model for driving a swing was an example of coupled pendula.
	In that case the rotation of the swinger was coupled to the rotation of the overall
	swing, which allows you to drive the swing.\\
	\indent In this lab we consider a different method for coupling pendula. Consider two
	pendula hanging from the same piece of horizontally$-$stretched rubber tubing.
	If one pendulum is held fixed and the second is displaced from equilibrium, the
	second one experiences a restoring torque from two separate sources: (a) gravity
	and (b) the rubber tubing. If both pendulums are displaced together each one
	experiences gravity and restoring torque from the tubing, but the tubing between
	the two plays no role because they both twist it in the same direction. But if the
	pendulums are displaced in opposite directions then the tubing between them
	is flexed, causing an extra restoring torque. This difference in restoring force
	between the \textquotesingle\textquotesingle together\textquotesingle\textquotesingle and \textquotesingle\textquotesingle opposite\textquotesingle\textquotesingle motions is the cause of the two slightly
	different frequencies that produce the beating you will see throughout this lab.\\
\section{Coupled Equations of Motion via Lagrangian Dynamics}
	For small displacements, the effect of gravity (plus a small contribution from the
	tubing) can be modeled as restoring torsional springs with spring constants $\kappa 1$
	and $\kappa 2$ for pendulum 1 and pendulum 2, respectively. This leads to a potential
	energy
\begin{equation}\label{13.1}
U=\dfrac{1}{2•}\kappa _1 {\theta _1}^{2}+\dfrac{1}{2}\kappa _2 {\theta _2}^{2}
\end{equation}
	The tubing also adds a coupling term to the potential energy of the form
\begin{equation}\label{13.2}
U_c=\dfrac{1}{2}\kappa _c(\theta _1 -\theta _2 )^{2}
\end{equation}
	Note that this extra restoring potential energy is zero if the angular displacements
	are equal. The total potential energy is
\begin{equation}\label{13.3}
	U=\dfrac{1}{2}\kappa _1 {\theta _1}^{2}+\dfrac{1}{2}{\kappa _2}{\theta _2}^{2} +\dfrac{1}{2}\kappa _c (\theta _1 - \theta _2)^{2}
\end{equation}
\subsection*{\small{P13.1}}\label{P13.1}
	Combine this potential energy function with the kinetic energy
\begin{equation}\label{13.4}
	T=\dfrac{1}{2}I _1 \dot{\theta _1}^{2} + \dfrac{1}{2} I_2\dot{\theta _2}^{2}
\end{equation}
\footnotetext{\textsuperscript{1}G. Fowles and G. Cassiday, Analytical Mechanics (Saunders, Fort Worth, 1999), p. 443$-$460.}
\end{minipage}
\newpage
\begin{parcolumns}[colwidths={2=0.25\linewidth}]{2}
\colchunk
{
to build the Lagrangian $(L = T  − U)$. Use the Lagrangian equation of motion
\begin{equation}\label{13.5}
\dfrac{\partial L}{\partial q_i} - \dfrac{d}{dt}(\dfrac{\partial L}{\partial \dot{q_i}})=0
\end{equation}
	\\to derive equations of motion for $\theta 1(t)$ and $\theta 2(t)$. Then simplify these
	equations by assuming that the two pendula are identical so that $\kappa _1 = \kappa _2 = \kappa$
	and I1 = I2 = I. Also eliminate the spring constants and moments of inertia
	in favor of frequencies according to the definitions
\begin{equation}\label{13.6}
	{\omega _0}^{2}=\dfrac{\kappa }{I}~~;~~{\omega _c}^{2}=\dfrac{\kappa _c}{I}
\end{equation}
	\\Finally, put these two second$-$order differential equations in coupled first
	order form.\\
	If you did \hyperref[P13.1]{P13.1} correctly, you should have arrived at the following four firstorder differential equations for describing the motion of the coupled$-$pendulum system:
\begin{equation}\label{13.7}
\dot{\theta _1}~~=~~\omega _1
\end{equation}
\begin{equation}\label{13.8}
	\dot{\theta _2}~~=~~\omega _2
\end{equation}
\begin{equation}\label{13.9}
	\dot{\omega _1} =-{\omega _0}^{2}\theta _1-{\omega _c}^{2}(\theta _1 - \theta _2)
\end{equation}
\begin{equation}\label{13.10}
	\dot{\omega _2} =-{\omega _0}^{2}\theta _2-{\omega _c}^{2}(\theta _2 - \theta _1)
\end{equation}
	\\The four variables are: the angular positions of the two pendulums$\theta _1(t)$ and
	$\theta _2(t)$ (remember that $\theta = 0$ corresponds to the pendulum hanging straight down)
	and, the angular velocities of the two pendulums $\omega _1(t) and \omega _2(t)$. The parameter
	$\omega _0$ is associated with the natural frequency of a single pendulum without any
	coupling, and the parameter $\omega _c$ is associated with the natural frequency of the
	middle section of tubing when attached to the two pendula.\\
\subsection*{\small{P13.2}}\label{P13.2}
\begin{enumerate}[label=\alph*]
\item  Use Mathematica to solve \hyperref[13.7]{Eq. 13.7}$-$(\hyperref[13.10]{13.10}) symbolically without
	initial conditions to see if you can find the two separate frequencies
	that cause the beating you will see when you solve them in Matlab.
\item Now numerically solve this system in Matlab with $\omega _0 = 1.3 ~and~ \omega _c =
	0.3$; for initial conditions let everything be zero except $\theta _1(0) = 0.3$.
	Make plots of $\theta _1(t) and \theta _2(t)$, one above the other using subplot\textsuperscript{2}
	like this:
\begin{lstlisting}[language=Matlab][numbers=none]
	subplot(2,1,1)
	plot(te,th1e)
	subplot(2,1,2)
	plot(te,th2e)
\end{lstlisting}
	Run long enough that something interesting happens, i.e., run at least
	long enough that $\theta _2(t)$ becomes large, then small again. You should
	be looking at the beat plot in \hyperref[Figure 13.1]{Fig. 13.1}. This pattern of increasing and
	decreasing amplitude in the plots of $\theta _1(t) ~and ~\theta _2(t)$ is an example of
	interference beats caused by the presence of two different frequencies
	in the dynamics.
\item  Run the solution out for long enough that when you take the FFT of
	$\theta _1(t)$ you can see the two peaks in the power spectrum corresponding
	to the two frequencies whose mixing causes the beats. Verify that
	the beat frequency $\omega _b = 2\pi /Tb$, (where Tb is the time for one of the
	oscillators to be at maximum amplitude, go to zero amplitude, then
	come back to maximum amplitude again) is related to the two peaks
	in the spectrum $\omega + ~and~ \omega −$ by
\begin{equation}\label{13.11}
	\omega _b = \omega _+ - \omega _ -
\end{equation}
	\\Also verify that the two frequencies you observe in the FFT are the two
	frequencies predicted by your Mathematica calculation.
	Note: the figure at the beginning of this lab does not have enough
	oscillations in it for the FFT to work well. As a general rule, your time
	plots should look solid if you want to use the FFT. A maximum time
	around 2000 works fine.
\item Now add a linear damping term to the equation of motion for pendulum number 2, start the system with these initial conditions: $\theta _1(0) =0.3,		\dot{\theta _1}=0,\dot{\theta _2}=0,\theta _2 = 0$ and study the motion of the two
oscillators. Use
\begin{equation}\label{13.12}
	\ddot{\theta _2} = \ \gamma \dot{\theta _2} + . . .
\end{equation}
	\\with $\gamma$ = 0.07. Look at the plots for $\theta _1(t) ~and~ \theta _2(t)$ and discuss what
	happens to the energy that was initially put into pendulum number 1.
\item Now drive pendulum number 1 by applying a small negative torque
	N1 = $−$0.3 whenever $\theta _1$ is positive and $\dot{\theta}$
	1 is negative. You will need
	to use an if statement in the M$-$file that defines the right$-$hand side
	of your set of differential equations to make this work. This driving
	force is similar to the escapement in a pendulum clock in which a
	mechanical linkage allows the weights to push on the pendulum when
	it is at the proper place in its motion. Think about this drive and verify
	that it always puts energy into pendulum number 1.
	As in part (b), don\textquotesingle t damp pendulum number 1; just keep the damping
	in pendulum number 2. Run the code long enough that the system
	comes to a steady state in which both pendulums have constant amplitude. Discuss the flow of energy in this system.
\end{enumerate}
}
\colchunk{
\begin{wraptable}{l}{4cm}
	\begin{tabular}{m{4cm}}
 This driving force is essentially the same as the intermittent torque you apply to
someone when you push
them in a swing.
\end{tabular}
\end{wraptable}

}
\end{parcolumns}
\begin{parcolumns}[colwidths={2=0.25\linewidth}]{2}
\colchunk{
\section{Coupled Wall Clocks}
	Now we are ready to study a very famous problem in dynamics. In the 1600s
	Christian Huygens observed that when two clocks are hung next to each other
	on a wall, they tend to synchronize with each other. Let\textquotesingle s see if we can make our
	equations of motion do this.
\subsection*{\small{P13.3}}\label{P13.3}
\begin{enumerate}[label=\alph*]
\item Begin by making both pendulums be damped and driven as described
	in \hyperref[P13.2]{P13.2}(d) and (e), but remove the coupling by setting $\omega _cc = 0$. Run
	the code and make sure that each clock comes to its own independent
	steady state.
	Now add weak coupling between the two by setting $\omega _c = 0.3$ again
	(the slight pushes and pulls that each clock exerts on the wall is the
	source of this coupling) and see if the clocks ever synchronize with
	each other. (Synchronization means that the two pendulums have the
	same period with some definite phase shift between them. This effect
	is called entrainment in the nonlinear dynamics literature).
	When you do these runs, start pendulum 1 with$ \theta _1 = 1~ and~ \dot{\theta _1}
	1 = 0$ and
	try various choices for the initial conditions of pendulum number 2.
	When they become entrained, check the phase difference between
	the two clocks. (A visual inspection is probably sufficient). In your
	numerical experiments, how many different phase relationships do
	you observe? (Try overlaid plots of $\theta _1 and \theta _2$ to see the phase relationships). Do your in$-$phase and out$-$of$-$phase entrained states 	have the
	same frequencies?
\item  According to the nonlinear dynamics literature, entrainment is an
	effect that depends on the oscillators being damped, driven, and
	nonlinear. Where is the nonlinearity in our equations of motion?
	\item Finally, let\textquotesingle s make these clocks a little more realistic by (i) replacing
	$-\omega ^{2}\theta by -\omega ^{2}sin\theta$ in each equation of motion and by (ii) having their
	natural frequencies be slightly different. Do this by changing $\omega _2$ in the equation of motion for pendulum 2 to
	$1.03\omega ^{2}, 1.1\omega ^{2}, and 1.25 \omega ^{2}$ (do
	all three cases). You should find that entrainment is relatively robust,
	meaning that the clocks don\textquotesingle t have to have exactly the same period to
	synchronize, but that if they are too different the effect is lost. Does
	this robustness depend on the strength of the coupling parameter $\omega _c$ ?
	Comment on what your answer to this last question has to do with
	real clocks on a wall.
\end{enumerate}
\section{Solving Nonlinear Equations}
	\subsection*{\small{P13.4}}\label{13.4}
		Read and execute the examples in \textit{Introduction to Matlab}, Chapter 14. Then
		complete the following exercises.
	\begin{enumerate}[label=\alph*]
			\item  Write a loop that makes an array containing the first 40 zeros of the
			Bessel Function $J_0(x)$. Find these zeros by writing a loop to load them
			using Matlab\textquotesingle s fzero command. You will have to give fzero a search
			range instead of just an initial guess, and this will be easier if you
			remember that the zeroes of $J_0(x)$ are separated by about $\pi$.
			Look through your list of zeros and make sure that there are no repeated values. Then plot $J_0(x)$ and put a red x at every zero.
			\item Solve the following set of equations using Matlab\textquotesingle s fsolve command 
			$$x^{2}+y^{2}+z^{2}=139$$
			$$\dfrac{x}{x+y-z}=3$$
			$$x\sqrt{z}=(10-y)^{2}$$
	\end{enumerate}
}
\footnotetext{\textsuperscript{2}For more information about subplots, look it up via help subplot using online help in Matlab.}
\colchunk{
\begin{wrapfigure}{r}{7cm}
	\captionsetup{singlelinecheck=off, margin={3.67cm, 0cm}, justification=raggedright, format=hang}
	\caption{ Energy passes back
and forth between $\theta _1$ and $\theta _2$ due
to beating.}\label {13.1}
	\includegraphics[scale=0.8]{ch131313133}
\end{wrapfigure}
}
\end{parcolumns}

\end{document}