\documentclass[../main.tex]{subfiles}
\begin{document}
\begin{parcolumns}[colwidths={2=0.3\textwidth}]{2}
\chapter{Visualizations and Qualitative Analysis}
\colchunk{
\begin{flushleft}
	Let\textquotesingle s start with some review of loops, logic, and matrices.
\subsection*{\small{P2.1}}\label{P2.1}
	Write a Matlab script that defines the following array
\begin{lstlisting}[language=Matlab][numbers=none]
	A=[14,42,91,79,95,65,3,84,93,67,75,74,39,65,17]
\end{lstlisting}
	and then performs a \textquotesingle\textquotesingle bubble sort\textquotesingle\textquotesingle to order the array elements in A from
	smallest to largest. A bubble sort is a simple sorting algorithm that works
	by repeatedly looping through an array using a for loop, comparing each
	pair of adjacent items and swapping them if they are in the wrong order.
	The for is nested inside a while that repeats until no swaps are needed in
	the for loop. The algorithm gets its name from the way smaller elements
	\textquotesingle\textquotesingle bubble\textquotesingle\textquotesingle to the top of the list. Step through your code using the debugging
	commands while watching the values in A to make sure it is doing what you
	think it should. To pass this exercise off, you must have at least 6 comment
	lines in your code.
\section{Line Plots}
	Matlab has a wealth of visualization tools available to help you view your data.
	Let\textquotesingle s look at some of the 1$-$dimensional plotting tools.
\subsection*{\small{P2.2}}\label{P2.2}
 	Read and work through \textit{Introduction to Matlab}, Chapters 5. Type and
	execute all of the material written in this kind of font and execute the
	examples. Then complete the following exercises.
\begin{enumerate}[label=\alph*]
\item Make a graph of $f(x) = x sin(x)$ from $x = 0 ~to~ x = 4\pi$. Label the axes
	and give the plot a title. Then overlay on the same frame a plot of
	$cos(x)$ and add a legend to the plot that identifies each curve. Use
	Matlab help for the legend command to learn how to do this.
\item Write a script with a for loop that calculates the first 20 terms of the
	recursion relation
	$$a_1=1 ; a_{n+1}=(\dfrac{n}{(n-\dfrac{1}{2})(n+\dfrac{1}{2})})a_n.$$
	And stores each value in an array a. Use Matlab\textquotesingle s debugging commands to step through your code while you watch the values change
	in the workspace window. (HINT: $a_{20} \approx 1.3*10^{-17}$)
	When you are sure your program is working correctly, add some code
	to plot the values of $a_n$ versus n using semilogy. Then overlay plots
	of $e^{−n}$ and 1/n! and label each line with a legend. Which function best
	matches the way the an terms fall off with n? You must have at least
	four comment lines in your code to pass it off.
\item Define an array x that contains the values from $x = 0 ~to~ x = 5$ with a
	step size $\Delta x = 0.01$. Make an empty array f the same size as x using
	the zeros command. Then use a for loop and logic commands to
	load f with the values:
\begin{equation}\label{2.1}
	f(x) = \left\{\begin{array}{lr}
       e^{x}, & for~0\leq x< 1\\
      e\times cos(x-1), & for~  1\leq n\leq 5
      \end{array}\right\}
\end{equation}
	Finally, plot $f (x)$ vs. x and label your axes. You must have at least four
	comment lines in your script to pass it off.
	HINT: For this problem do not use a command like
\begin{lstlisting}[language=Matlab][numbers=none]
	if x < 1
\end{lstlisting}
	because x is an array, not a single number. You will need to address
	individual elements of the arrays when you do your logic tests and
	assignment statements.
\end{enumerate}

\section{How does a differential equation make a curve?}
	Our purpose in this course is to analyze problems with differential equations.
	Before becoming reliant on numerical ODE solvers, you need to develop an
	intuition for how differential equations behave. You will need this intuition to
	propose and refine mathematical models for physical processes and have a sense
	of whether the solutions that a computer spits out are reasonable. If you don\textquotesingle t
	develop good intuitive skills, the many differential equations you\textquotesingle ll encounter in
	your physics courses will appear mysterious to you.
	Let\textquotesingle s look at a simple differential equation and try to translate it into words:
\begin{equation}\label{2.2}
	\dfrac{d}{dt}y=y
\end{equation}
	Since 
	$\dfrac{d}{dt}y$ is the slope of the function y(t) this differential equation says that the
	bigger y gets the bigger its slope gets. Let\textquotesingle s consider the two possible cases for
	initial conditions.
\paragraph*{Case 1: y(0)>0}
\begin{itemize}
	\item The differential equation then says that the slope is positive, so y is increasing. But if y increases its slope increases, making y increase more, making
	its slope increase more, etc. So the solution of this equation is a function
	like $e^{t}$ that gets huge as t increases
\end{itemize}
\end{flushleft}
}
\colchunk
{
\begin{wraptable}{l}{4cm}
	\begin{tabular}{m{4cm}}
The bubble sort is not an efficient way to sort. Matlab\textquotesingle s
sort command is much better, but we are learning how
to program here.
\end{tabular}
\end{wraptable}
}
\end{parcolumns}
\begin{parcolumns}[colwidths={2=0.3\textwidth}]{2}
\colchunk
{
\paragraph*{Case 2: y(0)<0}
\begin{itemize}
\item Now the differential equation says that the slope is negative, so y will have
to decrease, i.e., become more negative than it was at $t = 0$. But if y is more
negative then the slope is more negative, making y even more negative, etc.
Now the solution is a strongly decreasing function like $−e^{t}$
\end{itemize}
Now consider another example. Suppose that you have discovered some
process in which the rate of growth of the quantity y is not proportional to y itself,
as in exponential growth, but is instead proportional to some power of y,
\begin{equation}\label{2.3}
\dfrac{d}{dt}y=y^{p}
\end{equation}
This idea is referred to as \textquotesingle\textquotesingle explosive growth.\textquotesingle\textquotesingle Keeping in mind that with $p = 1$
we get the exponential function, this equation says that if y starts out positive,
y should increase even more than it did before, i.e., get bigger faster than the
exponential function. That would have to be pretty impressive, and it is$—$y goes
to infinity before t gets to infinity. \hyperref[Figure 2.1]{Fig. 2.1} shows a plot of the explosive growth
function for the cases of $P = 2$ and $P = 3$.
\hyperref[Figure 2.1]{Figure 2.1} The explosive growth
function defined by \hyperref[2.3]{Eq. 2.3} for two
values of P.
You can play this qualitative analysis game with second$-$order differential
equations too. Let\textquotesingle s translate the simple harmonic oscillator equation
\begin{equation}\label{2.4}
\dfrac{d^{2}}{dt^{2}}y=-y
\end{equation}
into words. We need to remember that the second derivative means the curvature
of the function: a positive second derivative means that the function curves
like the smiley face of someone who is always positive, while negative curvature
means that it curves like a frowny face. And if the second derivative is large in
magnitude then the smile or frown is very narrow, like a piece of string suspended
between its two ends from fingers held close together. If the second derivative is
small in magnitude it is like taking the same piece of string and stretching your
arms apart to make a wide smile or frown.\\
So what does \hyperref[2.4]{Eq. 2.4} say if $y = 1$ and $y\textquotesingle = 0$ to start? The first derivative is
zero, so y(t) comes out flat, and the second derivative is negative, so the function
curves downward, making y smaller, which makes the frowniness smaller, but
still negative, so y keeps curving downward until it crosses $y = 0$. Then with y
negative the differential equation says that the curvature is positive, making y
start to smile and curve upward. It doesn\textquotesingle t curve much at first because y is pretty
small in magnitude, but eventually y will have a large enough negative value
that y(t) turns into a full$-$fledged smile, stops going negative, and heads back
up toward $y = 0$ again. When it gets there y becomes positive, the function gets
frowny and turns back around toward $y = 0$, etc. So the solution of this equation
is an oscillation, cos(t) or sin(t).
\subsection*{\small{P2.3}}\label{P2.3}
For each of the following cases, use qualitative analysis to sketch the solution of the equation on paper.
\begin{enumerate}[label=\alph*]
\item
\begin{equation}
\dfrac{d}{dt}y=y^{2}~~~with~~~y(0)=-1
\end{equation}
\item 
\begin{equation}
\dfrac{d^{2}}{dt^{2}}y=y~~~with~~~y(0)=1~~~and~~~\dfrac{d}{dt}y(0)=0
\end{equation}
\end{enumerate}
\subsection*{\small{P2.4}}\label{P2.4}
Don\textquotesingle t start this problem until after making all of your sketches in \hyperref[P2.3]{P2.3}.
\begin{enumerate}[label=\alph*]
\item Verify, on paper, that the analytic solution to \hyperref[P2.3]{P2.3}(a) is $y(t) =   - 1/(1+ t)$.
Use Matlab to plot this function and compare it to your sketch.
\item Verify that the analytic solution to \hyperref[P2.3]{P2.3}(b) is $y(t)=(e^{-t}+e^{t})/2$ Use
Matlab to plot this function and compare it to your sketch.
\end{enumerate}
}
\colchunk
{
\begin{wrapfigure}{r}{9cm}
\captionsetup{singlelinecheck=off, margin={0cm, 4cm}, justification=raggedright, format=hang}
\caption{ The explosive growth
function defined by \hyperref[2.3]{Eq. 2.3} for two
values of P.}
\label {Figure 2.1}
\includegraphics[width=3cm]{ch2zd}
\end{wrapfigure}
}
\end{parcolumns}
\end{document}
