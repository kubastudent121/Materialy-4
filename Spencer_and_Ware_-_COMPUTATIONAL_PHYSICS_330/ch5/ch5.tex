\documentclass[../main.tex]{subfiles}
\begin{document}
\chapter{Playing Baseball with ODEs}
\begin{minipage}[t]{10cm}
In the previous lab, we learned a crude method for numerically solving differential equations called Euler\textquotesingle s method. In this lab we learn how to take the
next step in refining that rudimentary technique into a more accurate ODE solver.
After we have a good idea how ODE solvers are built and refined, we will introduce
you to some powerful differential equation solvers built into Matlab.\\
\section{Numerically Solving Differential Equations}
\subsection*{\small{P5.1}}\label{P5.1}
Read and work through Introduction to Matlab, Sections 9.1$-$9.2. Type and
execute all of the material written in this kind of font. Then work the
following problems.
\begin{enumerate}[label=\alph*]
\item Let\textquotesingle s start simple by modeling an object dropped from rest 10 m above
the ground. Neglect air resistance, so that the gravitational force is
simply $F = mg$ . On a piece of paper, write down Newton\textquotesingle s second
law for this system and then convert it to a first-order set of coupled
equations. Write the derivatives as finite differences on a grid in time
like we did in the last lab, and solve the resulting algebra equations
to derive the equations for Euler\textquotesingle s method. (Don\textquotesingle t peek at the answer,
derive Euler\textquotesingle s method again from scratch.)
\item  Implement your equations from part (a) in a Matlab script to solve the
differential equation. Keep track of all the values, and plot $y(t)$ until
your object hits the ground. Overlay a plot of the analytic solution and
compare these plots for various values of $\tau$.
\item Modify your code from part (b) to use second-order Runge$-$Kutta.
Evaluate how your accuracy changes as you vary $\tau$, and overlay plots
of the Euler\textquotesingle s method solution, the Runge$-$Kutta solution, and the
analytic solution. Compare the answer for various values of $\tau$.
\end{enumerate}
\subsection*{\small{P5.2}}\label{P5.2}
Read and work through \textit{Introduction to Matlab}, Sections 9.3.
\\
\textbf{Important:} As you work through this material in \textit{Introduction to Matlab},
you will learn how to use an M$-$file named $rhs.m$ to solve differential equations. As you do this problem and in later labs, please don\textquotesingle t keep using
the name $rhs.m$ over and over. Invent a unique name, like $rhs5 \_{} 2.m$, and
change the call to $ode45$ to correspond: $ode45(@rhs5 \_{} 2,...)$. This will
make it possible for you to come back later and see how you did each of the
problems.
\end{minipage}
\newpage
\begin{parcolumns}[colwidths={2=0.3\textwidth}]{2}
\colchunk{
\begin{enumerate}[label=\alph*]
\item  Use Matlab\textquotesingle s numerical differential equation solver $ode45$ to solve the
motion of the particle dropped from rest at a height of 10 m (no air
friction) by writing a rhs function and using Matlab\textquotesingle s $ode45$.
\item Make another version of your code from part (a) that uses an anonymous function instead of an external m$-$file rhs function. In subsequent problems and labs, you are free to use either syntax, but when
the functions get complicated its usually easier to use the external rhs
function.
\end{enumerate}
\subsection*{\small{P5.3}}\label{P5.3}
Use Matlab\textquotesingle s ode45 to numerically solve the following equation.
\begin{equation}\label{5.1}
\dfrac{dy}{dt}=ysint~~~~;~~~~y(0)=1
\end{equation}
Plot the numerical solutions from $t = 0$ to $t = 100$ and overlay a plot of the
analytic solution
$$y(t)=e^{1-cos(t)}$$
Fiddle RelTol and get a feel for how the accuracy changes with this parameter. Note: this differential equation is only first order, so you won\textquotesingle t have
u(1) and u(2) this time. Think carefully about how to change your code
from \hyperref[P5.2]{P5.2} to do this first$-$order problem.
\section{Baseball}
In Physics 121 you did the problem of a hard$-$hit baseball, but because you did
it without air friction you were playing baseball on the moon. Let\textquotesingle s play ball in a
real atmosphere now. The air$-$friction drag\textsuperscript{1} on a baseball is approximately given
by the following formula
\begin{equation}\label{5.2}
F_drag=-\dfrac{1}{2}C_d\rho _air\pi a^{2}|v|v
\end{equation}
where $C_d$ is the drag coefficient, $\rho$ air is the density of air, a is the radius of the ball,
and v is the vector velocity of the ball. The absolute value in \hyperref[5.2]{Eq. 5.2} pretty much
guarantees that we won\textquotesingle t find a formula for the solution of this problem, but that\textquotesingle s
fine since we know how to numerically solve differential equations now.
There are two forces acting on a baseball: air drag and gravity. Using Newton\textquotesingle s
second law $m\overrightarrow{r} =\sum F$ we see that equation of motion for the ball is
\begin{equation}\label{5.3}
	m\ddot{r} =F_drag-mg\overrightarrow{y}
\end{equation}
where r is the vector position of the ball, m is the mass of the baseball, g is the
acceleration of gravity, and we have chosen the $\overrightarrow{y}$ direction to be up. Since this is
a vector equation, it represents a system of equations$—$one for each dimension.
To simplify our life, let\textquotesingle s consider the motion to be just in the x$-$y plane with $\overrightarrow{x}$ as
the horizontal direction. Using the definition of velocity, we can convert \hyperref[5.3]{Eq. 5.3}
into the following set of four coupled first$-$order equations
\begin{equation}\label{5.4}
\dfrac{dx}{dt}=v_x~~~~\dfrac{dv_x}{dt}=-\dfrac{C_d	\rho _{air}\pi a^{2}v_x \sqrt{{v_x}^{2}{v_y}^{2}}}{2m}
\dfrac{dx}{dt}=v_x~~~~\dfrac{dv_x}{dt}=-\dfrac{C_d	\rho _{air}\pi a^{2}v_x \sqrt{{v_x}^{2}{v_y}^{2}}}{2m}
\end{equation}
\subsection*{\small{P5.4}}\label{P5.4}
\begin{enumerate}[label=\alph*]
\item  Use Matlab\textquotesingle s ODE solver to solve the set of equations (\hyperref[5.4]{5.4}) for a baseball with the following parameters:
\begin{equation}
C_d=0.35~~a=0.037m~~g=9.8m/s^{2}~~\rho _{air}=1.2kg/m^{3}~~m=0.145kg
\end{equation}
Put the point of contact between bat and ball at the origin $(x(0) = 0,
y(0) = 0)$. Write your initial conditions in terms of the initial angle $\theta$
and velocity $v_0$ of the baseball (i.e. $v0_x = v0$ $cos\theta $, $v0_y = v_0 ~sin\theta$) so we
can play with the angle and initial speed.
Plot y(t) and x(t) for the initial conditions of $\theta$ = 45\textdegree
and $v_0 = 60 m/s$.
Then plot the trajectory y(t) vs. x(t).
\item Once you have your plot for the trajectory in air, overlay the trajectory
that the ball would have experienced without air drag on the same
plot. Estimate the difference in range caused by air friction.
\item Power hitters say they would rather play in Coors Field in Denver than
in sea$-$level stadiums because it is so much easier to hit home runs.
Do they know what they are talking about?To find out, repeat part (a),
but instead of overlaying the no air friction plot, overlay the trajectory
of a ball hit in Denver and see if the ball goes significantly farther. 
The
density of air in Denver is about 15\%{} lower than it is at sea level.
\end{enumerate}
}
\colchunk
{
\begin{wrapfigure}{r}{9cm}
\captionsetup{singlelinecheck=off, margin={4cm, 0cm}, justification=raggedright, format=hang}
\caption{  The trajectory for a
home run hit, including the effect of air friction. Note that the
path is not a parabola.}
\label {Figure 5.1}
\includegraphics[width=4cm]{ch5zd}
\end{wrapfigure}
}
\end{parcolumns}
\footnotetext{\textsuperscript{1}For more information about the subject of air drag see R. Baierlein, Newtonian Dynamics
(McGraw Hill, New York, 1983), p. 1-7, and G. Fowles and G. Cassiday, Analytical Mechanics
(Saunders, Fort Worth, 1999), p. 55-65.}
\end{document}