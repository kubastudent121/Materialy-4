\documentclass[../main.tex]{subfiles}
\begin{document}
\chapter{The Harmonic Oscillator and Resonance}
\begin{minipage}[t]{10cm}
The harmonic oscillator is probably the most studied system in dynamics. In
this lab we use the numerical tools that we have developed to explore some of
the behavior of this system\textsuperscript{1}. Before we dive into the computational details, let\textquotesingle s
remind ourselves of the basic physics of a harmonic oscillator.
\section{The Basic Oscillator}
The basic oscillator equation is given by
\begin{equation}\label{6.1}
\dfrac{d^{2}}{dt^{2}}x(t)=-\omega ^{2}_0 x(t)
\end{equation}
The solutions to this equation are just sines and cosines that wiggle forever in
time with angular frequency $\omega _0$:
\begin{equation}\label{6.2}
x(t)=Asin(\omega _0t)+Bcos(\omega _0t  )
\end{equation}
or equivalently
\begin{equation}\label{6.3}
x(t)=Asin(\omega_0t+\varphi) 
\end{equation}
\subsection*{\small{P6.1}}\label{P6.1}
\begin{enumerate}[label=\alph*]
\item Sketch a phase space diagram for the harmonic oscillator by hand
on paper. Draw at least two phase$-$space curves for initial conditions
$x(0) = 1, ~v(0)~ = 0$ and another for $x(0) = 0~ and~ v(0) = 1$.
\item Use Matlab to make a phase$-$space plot of x(t) and v(t) for a simple
harmonic oscillator with $\omega _0 = 2.$
\item  Solve \hyperref[6.1]{Eq. 6.1} numerically using Matlab\textquotesingle s ODE solver with initial values $x_0 ~= ~1~ and~ v_0 ~=~ 0,~ v_0 ~=~ 1.4,~ and ~v_0~ =~ −1$ and run them from $t = 0 ~to~ ~t = 1$. Overlay the three trajectory plots on your flow plot. Identify
each of the three initial conditions on your plot, and explain what the
harmonic oscillator does along each trajectory.
\end{enumerate}
Of course, no real oscillator wiggles forever. To model a real system we need to
add damping.
\end{minipage}
\footnotetext{\textsuperscript{1}You can read more about the simple harmonic oscillator in the following references: R. Baierlein,
Newtonian Dynamics (McGraw Hill, New York, 1983), Chap. 2, and G. Fowles and G. Cassiday,
Analytical Mechanics (Saunders, Fort Worth, 1999), Chap. 3. }
\newpage
\begin{flushright}
\begin{minipage}[t]{10cm}
\section{The Damped Oscillator}
If we add some linear damping to the system, the harmonic oscillator equation
becomes
\begin{equation}\label{6.4}
\dfrac{d^{2}}{dt^{2}}x(t)=-{\omega _0}^{2}x(t)-2\gamma  \dfrac{d}{dt}x(t)
\end{equation}
where the damping factor $\gamma $ describes the amount of damping$—$a large $\gamma $ means
that there is a lot of damping. If you ask Mathematica to solve \hyperref[6.4]{Eq. 6.4}, it will tell
you that the solution is
\begin{equation}\label{6.5}
x(t)=A{e}^{-t(\gamma + \sqrt{\gamma ^{2}- {\omega _ 0} ^{2}})}+Be^{-t(\gamma - \sqrt{\gamma ^{2}-{\omega _0 }^{2}})}
\end{equation}
Equation (\hyperref[6.5]{6.5}) looks impressive, but if it\textquotesingle s supposed to be an oscillator that damps,
where are the sines and cosines? The problem is that we haven\textquotesingle t specified how
big $\omega _ 0$ and $\gamma$ are yet. Let\textquotesingle s think physically for a minute.\\
Suppose that you put a pendulum in motor oil at 50 degrees below zero. This
is an oscillator with a big $\gamma$. If you pull the pendulum back and release it, you
are not going to see any swinging; the pendulum will just slowly ooze back to
the vertical position and stay there. We refer to this system as being overdamped.
Look at \hyperref[6.5]{Eq. 6.5} and convince your lab partner that this solution is made up of
decaying exponentials when $\gamma$ is big (specifically $\gamma > \omega _ 0$).\\
Now imagine what would happen if we decrease the damping, say by warming
the oil up, or using WD$-$40 instead, or maybe even just air. In this case, the
pendulum will swing back and forth, but the amplitude will decrease over time.
But by what miracle did the exponential functions in the original solution become
sines and cosines? Recall Euler\textquotesingle s formula
$$
e^{i\theta }=cos(\theta ) + isin(\theta )
$$
which relates exponentials to wiggles through an imaginary argument. Note that
when $\gamma < \omega _0$
, the square$-$root in \hyperref[6.5]{Eq. 6.5} has a negative argument, and the square
root is imaginary. In this situation, we can rewrite \hyperref[6.5]{Eq. 6.5} as
\begin{equation}\label{6.6}
x(t) = e^{-t\gamma }[Ae^{-i\omega _d t}+Be^{i\omega _ d t}]
~~(when \gamma < \omega _ 0)
\end{equation}
where the frequency at which the damped oscillator \textquotesingle\textquotesingle wants\textquotesingle\textquotesingle to wiggle is given by
\begin{equation}\label{6.7}
\omega _ d = \omega _ 0 \sqrt{1-\gamma ^{2}/{\omega _ 0}^{2}}
\end{equation}
argument of this square root is positive $(\gamma > \omega _ 0 )$, then both of the fundamental solutions in \hyperref[6.5]{Eq. 6.5} are decaying exponentials and we only have damping
(no wiggles). The transition between the two is when the argument of the square
roots is zero, i.e., when $\gamma = \omega _ 0$
. This special case is called critical damping.
\end{minipage}
\end{flushright}
\newpage
\begin{minipage}[t]{10cm}
\subsection*{\small{P6.2}}\label{P6.2}
\begin{enumerate}[label=\alph*]
\item Use Matlab to make a phase$-$space plot for a damped harmonic oscillator with $\omega _ 0 = 2 ~and~ \gamma = 0.5.$
\item Solve \hyperref[6.4]{Eq. 6.4} numerically using Matlab\textquotesingle s ODE solver with initial values $x_0 = 1 ~and~ v_0 = 0, v_0 = 1.4,~ and~ v_0 = −1$, and run from $t = 0$ to
$t = 20$. Overlay the three trajectory plots on your flow plot. Identify
each of the three initial conditions on your plot, and explain what the
harmonic oscillator does along each trajectory.
\item  Change the damping coefficient to $\gamma = 4$ and repeat (a) and (b). Explain how the flow plot describes the overdamped system.
\item  An air$-$damped oscillator has damping more closely proportional to
the square of velocity rather than proportional to velocity. In equation
form, we write this as
\begin{equation}\label{6.8}
\dfrac{dx}{dt}=v~~;~~\dfrac{dv}{dt}=-{\omega _ 0}^{2}x-2\gamma v|v|
\end{equation}
epeat (a) and (b), but change your model to use the quadratic airdamping in \hyperref[6.8]{Eq. 6.8} with $\gamma = 0.5$ instead of linear damping. Explain
how this picture looks different from the ones in (a) and (b), and why.
\end{enumerate}
\section{The Driven, Damped Oscillator and Resonance}
If we add a sinusoidal driving\textsuperscript{2}
force at a frequency $\omega $ to the harmonic oscillator,
the equation of motion becomes
\begin{equation}\label{6.9}
\dfrac{d^{2}}{dt^{2}}x(t)=-{\omega _ 0}^{2}x(t)-2\gamma  \dfrac{d}{dt}x(t)+\dfrac{F_0}{m}cos(\omega t)
\end{equation}
Now we have two frequencies in play$—$the driving frequency $\omega $ and the dampedoscillator frequency $\omega _d$ given by \hyperref[6.7]{Eq. 6.7}. The typical behavior of the drivendamped harmonic oscillator starting from rest is as follows: an initial period of
start$-$up with some beating between the two frequencies ($\omega$ and $\omega _d$ ), then the
oscillations at $\omega _d$ damp out and the system transitions to a state of oscillation at
the driving frequency $\omega$ .
It is possible to solve \hyperref[6.9]{Eq. 6.9} symbolically, but let\textquotesingle s study its behavior numerically for practice.
\subsection*{\small{P6.3}}\label{P6.3}
Use Matlab to numerically solve \hyperref[6.9]{Eq. 6.9} and plot x(t) from $t = 0$ to $t = 300$
with $\omega _ 0 = 1$, $F_0 = 1$, $m = 1$, $\omega = 1.1$, and $\gamma  = 0.01$. Start from rest, with
$x(0) = 0$ and $x˙(0) = 0$. Note the initial beating between frequencies and
verify graphically that the final oscillation frequency of $x(t)$ is $\omega$.
\end{minipage}
\footnotetext{\textsuperscript{2}You can read more about the simple harmonic oscillator in the following references: R. Baierlein,
Newtonian Dynamics (McGraw Hill, New York, 1983), Chap. 2, and G. Fowles and G. Cassiday,
Analytical Mechanics (Saunders, Fort Worth, 1999), Chap. 3. }
\newpage
\begin{minipage}[t]{10cm}
\section{Resonance Curves}
When you push someone in a swing, you find that if you drive the system at the
right frequency, you can get large amplitude oscillations. This phenomenon is an
example of resonance, and the frequency at which the system has the maximum
response is called the resonance frequency $\omega _ r$
. If you drive a system at a frequency
far from $\omega _ r$ you only get small oscillations.
\subsection*{\small{P6.4}}\label{P6.4}
\begin{enumerate}[label=\alph*]
\item Make a new script by modifying your script from  so it makes a
plot of x(t) starting from rest and running for a long time so all the
beating has stopped. Use $F_0 = 1$, $m = 1$, $\gamma = 0.1$ $\omega _ 0 = 1$. Drive the
system at $\omega  = 1.1$. Then write some code that measures the amplitude
of the steady$-$state oscillations using the colon command to select
a few cycles of oscillation at the end of the time period and the max
command to find the maximum value within these oscillations.
\item  Add a for loop to your code in (a) that varies the driving frequency
from $4\omega = 0.5$ to $\omega = 1.5$ in steps of $\Delta \omega = 0.2$. For each driving frequency,
use your code to measure the steady$-$state oscillation amplitude A
(i.e. the amplitude of oscillation after all the beating has died out)
and make a plot of the steady$-$state amplitude A versus the driving
frequency $\omega$. Note the region where this curve has a maximum value
somewhere in the vicinity of $\omega _ d$ , but our resolution is too coarse to
see exactly where the maximum is.
\item  To better locate the maximum, modify your loop to look at the region
$\omega = 0.98$ to $\omega = 1.02$ with steps of $\Delta \omega = .001$. Find the frequency where
this curve has a maximum, and compare its location to $\omega _ d$ for this
system. Are they the same?
\end{enumerate}
In this problem you should note that the resonance frequency $\omega _ r$ (i.e. the peak of
the resonance curve) is not the same as the same as the damped frequency $\omega _ d$ .
When damping is small, $\omega _r$ and $\omega _ d$ are close, but they are not the same.
The plot you made in \hyperref[P6.3]{P6.3} is called a resonance curve. A resonance curve
plots the steady state oscillation amplitude (after the beating has died away)
vs. the driving frequency. You did this by brute force, but for the simple drivendamped equation we can find an analytic solution for the resonance curve. The
steady$-$state oscillation has the form
\begin{equation}\label{6.10}
x(t) = Acos(\omega t −\varphi )
\end{equation}
where A is the steady state amplitude, $\omega$ is the driving frequency, and $\varphi$ is the
phase difference between the driving force and the oscillator\textquotesingle s response. By
substituting \hyperref[6.10]{Eq. 6.10} into \hyperref[6.9]{Eq. 6.9} and analyzing the result, we find that the
steady$-$state amplitude A is given by
\begin{equation}\label{6.11}
A(\omega ) = \dfrac{F_ 0/m}{\sqrt{({\omega _ 0 -\omega ^{2}}^{2})^{2}+4\gamma ^{2}\omega ^{2}}}
\end{equation}
while the phase shift $\varphi$ is given by
\begin{equation}\label{6.12}
tan\varphi = \dfrac{2\gamma \omega }{{\omega _ 0} ^{2}-\omega ^{2}}
\end{equation}
\end{minipage}

\newpage
\begin{minipage}[t]{10cm}
\subsection*{\small{P6.5}}\label{P6.5}
\begin{enumerate}[label=\alph*]
\item  Write a matlab script that plots \hyperref[6.11]{Eq. 6.11} for the parameters in \hyperref[P6.4]{P6.4}
and compare the plot with your numerical results. Also plot $\varphi (\omega )$ and
describe to your lab partner what $\varphi$ represents.
\item Now make plots of A($\omega $) for several values of $\gamma$ and verify that a smaller
damping coefficient $\gamma$ leads to larger and sharper resonance.
\item Show analytically that the peak of the resonance curve A($\omega$) is not at
the damped frequency $\omega _ d$ , but occurs at
\begin{equation}\label{6.13}
\omega _ r=\sqrt{{\omega _d }^{2}-{\gamma}^{2}}=\sqrt{{\omega _ 0}^{2}-2{\gamma}^{2}}
\end{equation}
\end{enumerate}
HINT: Remember that to find the peak of a curve, you take its derivative and set it equal to zero.
\end{minipage}
\end{document}
