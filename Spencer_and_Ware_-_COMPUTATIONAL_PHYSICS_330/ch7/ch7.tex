\documentclass[../main.tex]{subfiles}
\begin{document}
\chapter{The Pendulum}
\begin{parcolumns}[colwidths={2=0.3\linewidth}]{2}
\colchunk{
The harmonic oscillator is an incredibly useful system to understand because
it is a reasonably good approximation to essentially every system that exhibits
oscillations, as long as the amplitude remains small. The classic example of a
oscillating system is a simple pendulum. In this lab we\textquotesingle ll study how the pendulum
resembles a harmonic oscillator and also how it differs.
\section{The Simple Pendulum}
\begin{flushleft}
The equation of motion of a simple pendulum is
\end{flushleft}
\begin{equation}\label{7.1}
\ddot{\theta }=-{\omega _ 0}^{2}sin\theta 
\end{equation}\\
where $\theta$ is the angle (in radians) between the pendulum and the vertical direction
and $\omega _0$ is the small$-$amplitude oscillation frequency. This is a nonlinear equation,
so we often use the small angle approximation $sin\theta$ $\approx$ $\theta$ to simplify \hyperref[7.1]{Eq. 7.1} into a
simple harmonic oscillator. But it doesn\textquotesingle t take a very large amplitude before the
small angle approximation falls apart. In this lab, we study the large amplitude
behavior of the pendulum, which can be quite different from the simple harmonic
oscillator.
\subsection*{\small{P7.1}}\label{P7.1}
 Show that \hyperref[7.1]{Eq. 7.1} is, in fact, nonlinear by showing that if you have two of
its solutions $\theta _ 1(t)$ and $\theta _2(t)$, then their sum $\theta _1(t)+\theta _2(t)$ \textit{is not} a solution of
the differential equation. When this happens, we say that the differential
equation is nonlinear. Use pencil and paper; Mathematica will just slow
you down.
\subsection*{\small{P7.2}}\label{P7.2}
 Use \hyperref[7.1]{Eq. 7.1} with Matlab to make a phase space diagram with Matlab\textquotesingle s
quiver and streamline commands. Use this diagram to describe the pendulum behavior for small oscillations, large oscillations, and motion where
the pendulum is rotating completely around rather than oscillating back
and forth. Identify trajectories for both clockwise and counter$-$clockwise
rotations.
\section{Period and Frequency of the Pendulum}
A pendulum is an extended object that is free to rotate with moment of inertia I
about a pivot point. The distance from the pivot point to the center of mass of the
object is $\ell $, and the small$-$amplitude oscillation frequency is $\omega _0 = \sqrt{mg\ell /I}$. If the
}
\colchunk
{
\begin{wrapfigure}{r}{9cm}
\captionsetup{singlelinecheck=off, margin={0cm, 4cm}, justification=raggedright, format=hang}
\caption{A phase space flow plot
for a pendulum.}
\label {Figure 7.1}
\includegraphics[width=4cm]{ch7zd}
\end{wrapfigure}
}
\end{parcolumns}
\newpage
\begin{parcolumns}[colwidths={2=0.3\textwidth}]{2}
\colchunk{
\begin{flushleft}
pendulum is a simple massless stick of length $\ell$ with all of the mass at the end of
the stick, the small$-$amplitude oscillation frequency simplifies to $\omega _ 0 =\sqrt{g/\ell }$
We can find the large$-$amplitude oscillation frequency of the pendulum by
using an energy method\textsuperscript{1} The kinetic energy of the pendulum is $I\dot{\theta}^{2}/2$ and the
potential energy is $mg\ell (1−cos\theta )$ (see \hyperref[Figure 7.2]{Fig. 7.2}). The total energy of a pendulum
can be found when the pendulum is at the maximum displacement, which we
will denote by $\theta _0$. At this point, the center of mass is at a height of $\ell (1−cos\theta _0)$
above the equilibrium position and the kinetic energy is zero, so the total energy
is $mg\ell (1 − cos\theta _0)$. As the pendulum oscillates, energy shuttles back and forth
between kinetic and potential according to
\begin{equation}\label{7.2}
\dfrac{1}{2}I\dot{\theta }^{2}+mg\ell (1-cos\theta ) =mg\ell (1-cos\theta _0)
\end{equation}\\
The first term on the left is the kinetic energy, the second term is the potential
energy, and the right side is the total energy of the system.
\subsection*{\small{P7.3}}\label{P7.3}
 Using paper and pencil, separate the variables $\theta$ and t in \hyperref[7.2]{Eq. 7.2} and show
that it can be written as
\begin{equation}\label{7.3}
\omega _0dt=\dfrac{d\theta }{\sqrt{2cos\theta -2cos\theta _0}}
\end{equation}\\
To find the period of oscillation, we integrate both sides of \hyperref[7.3]{Eq. 7.3} over a
quarter period of the motion (from $\theta = 0$ to $\theta = \theta _0$ on the angle side and from
$t = 0$ to $t = T /4$ on the time side), like this
\begin{equation}\label{7.4}
\omega _0\int_{0}^{T/4}dt=\dfrac{1}{\sqrt{2}}\int_{0}^{\theta _0} \dfrac{d\theta }{\sqrt{cos\theta -cos\theta _0}}
\end{equation}\\
The time integral on the left is simply $\omega _0 T /4$, but the $\theta$ integral on the right is
difficult. After carrying out the time integral and performing some judicious
variable substitutions and a little algebraic massaging, we can rewrite \hyperref[7.4]{Eq. 7.4} as
\begin{equation}\label{7.5}
T=\dfrac{4}{\omega _0}\int_{0}^{\pi /2}\dfrac{d\phi }{\sqrt{1-sin^{2}(\theta _0/2 sin^{2}\phi )} }
\end{equation}\\
The $\theta$ integral in \hyperref[7.5]{Eq. 7.5} is not any easier than the $\theta$ integral in \hyperref[7.4]{Eq. 7.4}, but it
has come up in enough problems that it has been given a name: the complete
elliptic integral of the first kind, called K(m):
\begin{equation}\label{7.6}
K(m)\equiv \int_{0}^{\pi /2}\dfrac{d\phi}{\sqrt{1-msin^{2}\phi}}
\end{equation}
\end{flushleft}
}

\colchunk
{
\begin{wrapfigure}{r}{9cm}
\captionsetup{singlelinecheck=off, margin={4cm, 0cm}, justification=raggedright, format=hang}
\caption{ A simple pendulum
comprised of a massless stick of
length $\ell$ with a mass m at the end.}
\label {Figure 7.3}
\includegraphics[width=4cm]{ch7zd2}
\end{wrapfigure}
\begin{wrapfigure}{r}{5.7cm}
\captionsetup{singlelinecheck=off, margin={1.2cm, 0cm}, justification=raggedright, format=hang}
\caption{ \small{The frequency of a pendulum depends on the amplitude
of oscillation. The variation of frequency with amplitude is smallest
for low$-$amplitude oscillations, so
its easier to get good accuracy with
long pendulum and small angle
oscillations as in a grandfather
clock.}}
\label {Figure 7.2}
\includegraphics[width=4cm]{ch7zd1}
\end{wrapfigure}
}
\end{parcolumns}
\begin{parcolumns}[colwidths={2=0.3\linewidth}]{2}
\colchunk{
	Matlab and Mathematica know how to evaluate K(m) functions for 0 $\leq m \leq 1$ just
	like they can evaluate sines, cosines, and Bessel functions. Thus, we can write the
	period T of the pendulum as
\begin{equation}\label{7.7}
	T=\dfrac{4}{\omega _0}K(sin^{2}(\theta _0/2))
\end{equation}
	Now we can use the relation $\omega = 2\pi /T$ to obtain an expression for the angular
	frequency of the pendulum as a function of amplitude $\theta _0$.
\begin{equation}\label{7.8}
	\omega(\theta _0)=\dfrac{\pi \omega _0}{2K(sin^{2}(\theta _0/2)}	
\end{equation}
	Note that the natural oscillation frequency $\omega (\theta _0)$ of the pendulum depends
	on amplitude $\theta _0$, as shown in \hyperref[Figure 7.4]{Fig. 7.4}. This gives the pendulum some interesting
	characteristics.
\subsection*{\small{P7.4}}\label{P7.4}
	Use Matlab to plot $\omega (\theta _0)$ from $\theta _0 = 0$ to $\theta _0 = \pi $ with $\omega _0 = 1$) and explain
	physically why it looks like it does. In particular, explain why the frequency
	goes to zero at $\theta _0 = \pi$. You\textquotesingle ll need to use the online help to see the syntax
	for evaluating the elliptic integral function.
\begin{flushleft}
	Now let\textquotesingle s solve the pendulum equation numerically using Matlab.
\end{flushleft}
\subsection*{\small{P7.5}}\label{P7.5}
	Use Matlab\textquotesingle s numerical differential equation solver ode45 to solve the pendulum, again with $\omega _0 = 1$ and initial conditions $\theta (0) = \theta _0$ and $\omega (0) = 0$. Plot
	the solution $\theta (t)$ for the following values of $\theta _0$: $0.1, 0.5, 1.0, \pi /2, 0.9\pi $,and
	0.98$\pi$ .For each case overlay a plot of a cosine function of matching amplitude and with a frequency $\omega (\theta _0)$ from \hyperref[7.8]{Eq. 7.8}. Verify that \hyperref[7.8]{Eq. 7.8}) gives
	the correct frequency, but that for large amplitudes the pendulum motion
	is not sinusoidal.
\subsection*{\small{P7.6}}\label{P7.6}
	Now let\textquotesingle s study what happens when we add driving and damping.
\begin{enumerate}[label=\alph*]
	\item First we\textquotesingle ll review what happens when we drive an undamped harmonic oscillator.Write a Matlab script that solves the driven oscillator equation
\begin{equation}\label{7.9}
	\ddot{y} + \omega ^{2}_0 y(t)=F_0sin(\omega t)
\end{equation}\\
	and plot the solution y(t) with $\omega _0 = \omega = 1$. Start from rest and run
	for a long enough time that you can see the amplitude heading off to
	infinity, even with small values of $F_0$.
	\item  Now drive an undamped pendulum with an external torque, like this
\begin{equation}\label{7.10}
	\ddot{\theta }+\omega^{2} _0sin\theta = \alpha sin \omega _\tau t.
\end{equation}\\
	Drive the pendulum at resonance for small amplitudes, with $\omega _0 = 1$,
$\omega _\tau = 1$, and $\alpha = 0.1$. Start at rest and run for a long enough time that
	you can see that the pendulum amplitude doesn\textquotesingle t simply go to infinity
	like the harmonic oscillator. Explain why not.
\item Finally, add some linear damping, to the pendulum equation like this:
\begin{equation}\label{7.11}
\ddot{\theta }+\omega^{2}_0sin\theta = \alpha sin\omega _\tau t -\gamma \dot{\theta}
\end{equation}\\
	Use $\gamma = 0.1$ and the same conditions as in (b) and watch how the
	motion changes. Explain the damped behavior and explore how
	it depends on $\alpha$. Also vary the driving frequency $\omega$ in the range
	$0.90\omega _0 \rightarrow 1.05\omega _0$ and explain why $\omega = \omega _0$ doesn\textquotesingle t give the largest
amplitude.
\end{enumerate}
}
\colchunk
{
\begin{wrapfigure}{r}{5cm}
\captionsetup{singlelinecheck=off, margin={1cm, 0cm}, justification=raggedright, format=hang}
\caption{\ref{Figure 7.4} Oscillation frequency
as a function of the maximum
amplitude $\theta _0$.}
\label {Figure 7.4}
\includegraphics[width=4cm]{ch7zd55}
\end{wrapfigure}
}
\end{parcolumns}
\footnotetext{\textsuperscript{1}G. Fowles and G. Cassiday, Analytical Mechanics (Saunders, Fort Worth, 1999), p. 318-320}
\newpage
\section{Differential Equations in Mathematica}
\begin{minipage}[t]{10cm}
	While the focus of this course is on learning numerical techniques in Matlab,
	Mathematica also has some excellent differential equation solving abilities that
	you should be aware of. Let\textquotesingle s take a break from Matlab and learn some of the
	basics in Mathematica.
\subsection*{\small{P7.7}}\label{P7.7}
	Read the section titled \textquotesingle\textquotesingle Symbolic solutions to ordinary differential equations\textquotesingle\textquotesingle in the Mathematica tutorial Differential equations with Mathematica
	(available on the Physics 330 course web page).
\subsection*{\small{P7.8}}\label{P7.8}
	 Use Mathematica to solve the following differential equations in general
	form (no initial conditions).
	\begin{enumerate}[label=\alph*]
	\item  Bessel\textquotesingle s Equation
		\begin{equation}
			x^{2}(\dfrac{d^{2}}{dx^{2}}f(x))+x(\dfrac{d}{dx}f(x))+(x^{2}-n^{2})f(x)=0
		\end{equation}
	\item  Legendre\textquotesingle s Equation
		\begin{equation}
				(1-x^{2})(\dfrac{d^{2}}{dx^{2}}f(x))-2x(\dfrac{d}{dx}f(x))+n(n+1)f(x)=0
		\end{equation}
	\end{enumerate}
\subsection*{\small{P7.9}}\label{P7.9}
	 Read the section titled \textquotesingle\textquotesingle Numerical solutions to ordinary differential equations\textquotesingle\textquotesingle in the Mathematica tutorial Differential equations with Mathematica
	\begin{enumerate}[label=\alph*]
		\item Ask Mathematica to solve the following differential equation symbolically and see what happens.
		\begin{equation}\label{7.12}
			\dfrac{d^{2}}{dx^{2}}y(x)=10sin(y(x))cos(x)
		\end{equation}
		Now write the equation as a first order set, and solve it numerically
	    with $y(0) = 0$ and $v(0) \equiv y\textquotesingle (0)+0.1 .$ 
	    Plot y(x) from $x = 0$ to $x = 100$.
	\end{enumerate}
\end{minipage}
\end{document}
