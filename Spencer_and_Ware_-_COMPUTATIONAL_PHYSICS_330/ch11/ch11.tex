\documentclass[../main.tex]{subfiles}
\begin{document}
\chapter{The Pendulum with a High Frequency Driving Force}
\begin{flushleft}
\begin{minipage}[t]{12cm}
	Consider\textsuperscript{1} an un$-$driven equation of motion of the form
\begin{equation}\label{11.1}
	\ddot{x}=-\dfrac{\partial V}{\partial x}.
\end{equation}
	For instance, a harmonic oscillator has $V(x) = kx^{2}/2m$
	and pendulum has $V (x) = - (g/L)cosx$. Let the characteristic time over which this system changes appreciably be the period T , e.g.$T=2\pi /\sqrt{g/L}$
	for the pendulum. We now drive this
	system with a very high frequency force that depends on both the particle position
	x and time t so that the equation of motion becomes
\begin{equation}\label{11.2}
	\ddot{x}=-\dfrac{\partial V}{\partial x}+A(x)sin\omega t
\end{equation}
	with
\begin{equation}\label{11.3}
	\omega >>2\pi /T
\end{equation}
	defining what we mean by high frequency. If we use our intuition (perhaps
	thinking about what it feels like to drive at high speed over a back$-$country dirt
	road that has developed wash boards) we might guess that the motion described
	by this differential equation would consist of some sort of slowly varying motion
	on the time scale T plus a high frequency low amplitude vibration at frequency $\omega$.
We make this guess precise by writing
\begin{equation}\label{11.4}
	x(t)=X(t)+\upxi (t),
\end{equation}
	where X(t) describes the slow motion (think about the car winding its way around
	curves and over hills) and $\upxi (t)$ describes the small amplitude high frequency oscillations (think about stuff in the glove compartment rattling, your teeth chattering,
	etc.). An example of this kind of motion is shown in \hyperref[Figure 11.1]{Figure 11.1}. The smooth curve is
	X(t) while the bumpy curve is $x(t) = X(t)+\upxi (t)$. The function $\upxi (t)$ is the difference
	between the two curves.
\section{Perturbation Theory}
	Because $\upxi (t)$ is caused by the sinusoidal driving force, we will see that its time
	average is zero, and the wide separation of time scales allows us to assume that
	(t) changes only slightly during one period of the high frequency motion. Substituting Eq. (\hyperref[11.4]{11.4}) into Eq. (\hyperref[11.2]{11.2}) and expanding in small $\upxi$ through first order
	gives
\begin{equation}\label{11.5}
	\ddot{X}+\ddot{\upxi}=-\dfrac{\partial V}{\partial x}|_{x=X} -\upxi\dfrac{\partial^{2}V}{\partial x^{2}}|_{x=X}+A(X)sin\omega t + \upxi\dfrac{\partial A}{\partial x}|_{x=X}sin\omega t
\end{equation}
	We first attack this equation by looking at the high frequency terms. The term
	$Asin\omega t$ is a big term, as is $\ddot{\upxi}$ because of its rapid variation in time $\ddot{\upxi}\approx -\omega^{2}\upxi with \omega large)$ All of the other high frequency terms are small compared to these two
	because $\upxi$ is small, so we have (approximately)
\begin{equation}\label{11.6}
\ddot{\upxi}=A(X)sin\omega t
\end{equation}
	with X approximately constant because it varies so slowly. A simple integration
	yields the rapidly varying position and velocity
\end{minipage}
\end{flushleft}
\footnotetext{\textsuperscript{1}This analysis is borrowed from Mechanics by Landau and Lifshitz: L. D. Landau and E. M.
Lifshitz Mechanics (Pergamon Press, New York, 1976), p. 93$-$95.
}
\newpage
\begin{flushright}
\begin{minipage}[t]{12cm}
\begin{equation}\label{11.7}
	\upxi (t)=-\dfrac{A(X)}{\omega^{2}}sin\omega t~~;~~\ddot{\upxi}(t)=-\dfrac{A(X)}{\omega}cos\omega t
\end{equation}
	We now substitute this result into Eq. (\hyperref[11.5]{11.5}) and time average every term in
	the equation over one period of the high frequency motion. Terms that contain
	single powers of $\upxi$, $cos\omega t$, or sinωt average to zero while in the last term, which
	contains $sin^{2} \omega t$, we may replace $sin^{2} \omega t$ by its time average of 1/2 to obtain
\begin{equation}\label{11.8}
\ddot{X}=-V\textquotesingle (X)-\dfrac{A(X)}{2\omega ^{2}}\dfrac{dA}{dX}
\end{equation}
	As you can see, the low frequency motion of the oscillator is altered by the presence of this rapidly oscillating force, provided that the force depends on X. This
	means that a simple high$-$frequency external force of the form $Asin\omega t$ with A
	constant has no effect on the slow motion.
	We are not quite finished because we haven\textquotesingle t discussed the initial conditions.
	Suppose that we have initial conditions
$$
	x(0)=x_0~~;~~\dot{x}(0)=v_0
$$
	Using Eq. (\hyperref[11.4]{11.4}) we have
$$
	X(0)+\upxi (0) = x_0 ~~; ~~\dot{X}(0)+ \dot{\upxi}(0) = v_0
$$
	which can be combined with Eq. (\hyperref[11.7]{11.7}) at $t = 0$ to obtain the proper initial conditions for the slow$-$motion variable X:
\begin{equation}\label{11.9}
	X(0)=x_0~~;~~\dot{X}(0)=v_0+A(x_0)/\omega
\end{equation}
	With this choice of initial conditions a combined plot of x(t) and X(t) shows that
	x(t) wiggles and slowly varies, while X(t) tracks right with it, but with all of the
	wiggles smoothed out.
	\section{Driven Pendulum}
	An interesting example of this kind of system is a pendulum whose support point
	vibrates rapidly up and down like this:
\begin{equation}\label{11.10}
	y_{support}=bsin\omega t
\end{equation}
	A simple way to find the new equation of motion of the pendulum is to use Einstein\textquotesingle s principle of equivalence between acceleration and gravity: If the support
	point is accelerating upward with acceleration $a_{support}$, then the pendulum will
	experience a downward gravitational force$ −ma_{support}$. Hence we may write for
	the effective acceleration of gravity acting on the pendulum $a_{support} = \ddot{y}_{support} =
	−\omega ^{2}bsin\omega t$ so that the total acceleration, including ordinary gravity is
\begin{equation}\label{11.11}
	g_{eff} = g − a_{support} = g − \ddot{y}_{support} = g +b\omega^{2} sin\omega t 
\end{equation}
	which then leads to the equation of motion
\begin{equation}\label{11.12}
	\ddot{\theta}=-{\omega_0}^{2}sin\theta - \dfrac{b\omega ^{2}}{L}sin\theta sine \omega t
\end{equation}
	where ${\omega _0}^{2} = g /L$. This equation of motion matches Eq. (\hyperref[11.2]{11.2}) if we write
\begin{equation}\label{11.13}
	A(\theta)=-\dfrac{b\omega ^{2}}{L}sin\theta
\end{equation}
	which then leads to the following slow time$-$averaged equation of motion [see
	Eq. (\hyperref[11.8]{11.8})]:
\begin{equation}\label{11.14}
	\ddot{\Uptheta}=-{\omega _0}^{2}sin\Uptheta-\dfrac{b^{2} \omega ^{2}}{2L^{2}}sin\Uptheta cos\Uptheta
\end{equation}
\end{minipage}
\end{flushright}
\newpage 
\begin{parcolumns}[colwidths={2=0.3\linewidth}]{2}
\colchunk{
\subsection*{\small{P11.1}}\label{P11.1}
\begin{enumerate}[label=\alph*]
	\item  Use Matlab\textquotesingle s ode45 to solve for the motion of a rapidly driven pendulum by solving both Eq. (\hyperref[11.12]{11.12}) and Eq. (\hyperref[11.14]{11.14}) with $\omega _0 = 1, ~L = 1,
	b = .02, and \omega = 30$ with initial conditions $\theta (0) = 1, ~\dot{\theta}(0) = 0$ and run
	for a total time of 30 seconds. Overlay the plots of $\theta (t)$ from both equations to see that the averaged solution approximates the un$-$averaged
	solution.
	Now run it again with initial conditions$ \theta (0) = 3.1,~ \dot{\theta}(0) = 0$ and check
	the agreement again.
	Note: The averaged solution won\textquotesingle t go through the middle of the wiggles of the full solution unless you adjust the averaged initial conditions as shown in Eq
	When you do it right your plot should
	look like \hyperref[Figure 11.1]{Figure 11.1}.
	\item  Now redo part (a) with everything the same except use $b = .05$ this time.
	You should be surprised, astounded, and amazed at what happens
	with $\theta (0) = 3.1$. This case is a nearly straight up pendulum, which
	should fall over, but as you can clearly see, the pendulum is now stable
	in the straight$-$up position. This is not a mistake, as you can discover
	by examining the electric saber$-$saw demonstration at the front of the
	room.
	\item Analyze this situation more carefully by finding the effective potential
	that produces the right$-$hand side of the slow equation of motion,
	Eq. (\hyperref[11.14]{11.14}), i.e., find $V (\Uptheta)$ such that
\begin{equation}\label{11.15}
	-\partial V/\partial \Uptheta = -{\omega _0}^{2}sin\Uptheta -\dfrac{b^{2}\omega ^{2}}{2L^{2}}sin\Uptheta cos\Uptheta
\end{equation}
	(integrate both sides of this equation to obtain $V (\theta )$). Then plot this
	potential from $\Uptheta = 0 to \Uptheta = 2\pi $ for various values of b in the range
	$b = 0~ to~ b = .1$ and notice what happens at $\Uptheta = \pi$ as b increases. Then
	use calculus to find the critical value of b at which the straight$-$up
	pendulum first becomes stable and use your code from part (b) to
	verify that this threshold value is correct.
\end{enumerate}
	These low frequency effective forces that arise from high$-$frequency non$-$linear
	effects are called \textit{ponderomotive forces} and they show up all the time in physical
	problems. This problem is just a small taste of a very large field.
}
\colchunk{
\\
\\
\\
\\
\\
\\
\\
\\
\\
\\
\\
\\
\\
\begin{wrapfigure}{r}{9cm}
\captionsetup{singlelinecheck=off, margin={0cm, 2cm}, justification=raggedright, format=hang}
\caption{}
\label {Figure 11.1}
\includegraphics[width=4cm]{gghd}
\end{wrapfigure} 
}
\end{parcolumns}
\end{document}