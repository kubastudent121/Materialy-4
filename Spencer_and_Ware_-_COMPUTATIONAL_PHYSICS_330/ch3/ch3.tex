\documentclass[../main.tex]{subfiles}
\begin{document}

\chapter{Phase Space and Matlab Functions}
\section{Surface and Flow Plots}
\subsection*{\small{P3.}}\label{P3.1}
\begin{parcolumns}[colwidths={2=0.3\textwidth}]{2}
\colchunk{
\begin{flushleft}
Read and work through \textit{Introduction to Matlab}, Chapter 6. Type and execute all of the material written in this kind of font and execute the
examples. Then do the following exercises:
\end{flushleft}
\begin{enumerate}[label=\alph*]
\item Write a script that makes a Matlab surface plot of the \textquotesingle\textquotesingle mountain\textquotesingle\textquotesingle
function \hyperref[Figure 3.1]{Fig. 3.1}:
\begin{equation}\label{3.1}
f(x,y)=e^{-\text{|}x-siny\text{|}}(1+\dfrac{1}{5}cos(\dfrac{x}{2})(1+\dfrac{4}{3+10y^{2}})
\end{equation}
Plot it from -5 to 5 in x and from -6 to 6 in y and add labels for the
x and y axes. Make sure the labels correspond to the correct axes. If
your plot is solid black, don\textquotesingle t use such a fine grid in x and y.
\item In a certain region of the atmosphere, the wind is blowing with velocity
that is constant in time, but varies spatially according to
\begin{equation}\label{3.2}
\dfrac{dx}{dy}=v_x=0.2x^{2}+0.5y^{2}+20~~~~
\dfrac{dy}{dt}=v_y=-0.1y^{3}+0.5x^{2}-10
\end{equation}
Write a script that makes a quiver plot of the wind velocity over the
region $-10$ to 10 for x and y. Now add some stream lines beginning on
the left edge of your plot using the streamline command as shown in
\hyperref[Figure 3.2]{Fig. 3.2}.\\
\end{enumerate}
\begin{flushleft}
The plot you created in \hyperref[P3.2]{P3.2}(b) is referred to as a flow plot. The arrows that
you produced with the quiver command show the magnitude and the direction
of the velocity at each point, and the streamlines show the path that a particle
would follow in this velocity field. You can also use this type of plot to understand
the behavior of differential equations.
\section{Phase Space}
\end{flushleft}
You can often visualize the solution of a second$-$order differential equation without actually solving it using phase space\textsuperscript{1}
techniques.In classical mechanics you will learn to call the two$-$dimensional plane defined by the variables q and $p =\dfrac{∂L}{∂q}$
phase space (L is the Lagrangian). But for simplicity, in this lab we will use the
position x and velocity v as the phase space variables.
A second order differential equation can always be separated into a set of
first-order equations by defining an intermediate variable. For instance, a onedimensional projectile with the constant acceleration is described by the differential equation
}
\colchunk{
\begin{wrapfigure}{l}{9cm}
\captionsetup{singlelinecheck=off, margin={0cm, 4cm}, justification=raggedright, format=hang}
\caption{The \textquotesingle\textquotesingle mountain\textquotesingle\textquotesingle function.}
\label {Figure 3.1}
\includegraphics[width=4cm]{latexch3}
\end{wrapfigure}\\
}
\end{parcolumns}
\footnotetext{\textsuperscript{1}R. Baierlein, Newtonian Dynamics (McGraw Hill, New York, 1983), p. 51$-$54, 140$-$144, and G.
Fowles and G. Cassiday, Analytical Mechanics (Saunders, Fort Worth, 1999), p. 93$-$98.}
\newpage
\begin{flushright}
\begin{minipage}[t]{10cm}
\begin{equation}\label{3.3}
\dfrac{d^{2}x}{dt^{2}}=-g
\end{equation}
By defining an intermediate variable v, this second-order differential equation
can be written as a system of first$-$order differential equations like this:
\begin{equation}\label{3.4}
\dfrac{dx}{dt}=v ~~and~~\dfrac{dv}{dt}=-g
\end{equation}
Notice that the position and velocity coordinates in \hyperref[3.4]{ Eq 3.4} have the same form
as the flow velocities in \hyperref[3.2]{ Eq 3.2}, i.e. the first derivatives on the left equal expressions on the right with no derivatives. If you think of dx/dt and dv/dt in \hyperref[3.4]{ Eq 3.4}
as flow velocities in the x-v plane, the right-hand sides of these equations tell
you what the \textquotesingle\textquotesingle flow\textquotesingle\textquotesingle velocity is at each point in space. At any point in time t, the
coordinate $[x(t), v(t)]$ gives the phase-space point that represents the \textquotesingle\textquotesingle state\textquotesingle\textquotesingle of
the system. Given an initial starting point, you can then trace out a curve called a
phase space trajectory analogous to the streamlines we plotted in the flow plot.
Part of the power of phase space flow plots is that you don\textquotesingle t have to solve the
differential equation to make the flow plot.\\ You just evaluate the right-hand sides
of \hyperref[3.4]{ Eq 3.4}, for example, and draw arrows at each point in the (x, v) space that
indicate which way the solution at that point will move if we take a small step in
time. To draw the phase space trajectories, we just connect up the arrows over
short time intervals (or let Matlab do it for with with streamline). In this way
you can explore the behavior of the system for a wide range of initial conditions
without ever actually solving the ODE for any of these conditions.
\subsection*{\small{P3.2}}\label{P3.2}
 Sketch a phase space diagram for the one-dimensional projectile in \hyperref[3.4]{ Eq 3.4}
by hand on paper. Then draw some phase-space trajectories for a ball being
thrown up with various velocities. After you have done your work by hand,
check it by using Matlab with the quiver and streamline tools.
\subsection*{\small{P3.3}}\label{P3.3}
\begin{equation}\label{3.5}
\dfrac{d^{2}x}{dt^{2}}=-x
\end{equation}
First sketch on trajectory by hand on a paper, then use Matlab to plot
the phase space diagram with many trajectories using the quiver and
streamline tools. Verbally describe the motion represented by each curve.
\end{minipage}
\end{flushright}
\newpage
\begin{parcolumns}[colwidths={2=0.3\textwidth}]{2}
\colchunk{
\begin{flushleft}
\subsection*{\small{P3.4}}\label{P3.4}
 Use Matlab to plot a phase space diagram for the angle of a rigid pendulum,
given by
\begin{equation}\label{3.6}
\dfrac{d^{2}\theta}{dt^{2}}=-sin(\theta)
\end{equation}
Play with the range of the plot until you can clearly see motions that wiggle back and forth, and others that just spin around like a propeller on a
plane. Identify curves that are clockwise spinning, curves that are counterclockwise spinning, and curves that wiggle back and forth.
\subsection*{\small{P3.5}}\label{P3.5}
 Use qualitative analysis to sketch the solution of the equation
\begin{equation}
\dfrac{d^{2}\theta}{dt^{2}}y=-y^{2}~~with~~y(0)=1~~and~~\dfrac{d}{dt}y(0)=0
\end{equation}
on paper like we did in the last lab. This ODE doesn\textquotesingle t have an analytic
solution. Make a phase$-$space quiver plot and use streamline to overlay
a phase$-$space trajectory corresponding to the correct initial conditions and
compare this trajectory with your sketched solution and make sure they are
consistent.
\section{Functions in Matlab}
To this point, we\textquotesingle ve mostly relied on Matlab\textquotesingle s built$-$in functions tied together with
some code to perform our work. As our numerical techniques advance, we\textquotesingle ll need
to be able to write our own functions. Pay close attention to this material, because
it will be important throughout the remainder of the course.
\subsection*{\small{P3.6}}\label{P3.6}
 Read and work through \textit{Introduction to Matlab}, Chapter 7. Type and execute
all of the material written in this kind of font. Then do the following
exercises:
\end{flushleft}
\begin{enumerate}[label=\alph*]
\item Write an m$-$file function called EulerSum.m that computes the quantity
\begin{equation}
S_e(N)=(\sum_{n=1}^{N}\dfrac{1}{n})-ln(N)
\end{equation}
You can either use a loop or the sum command to compute the sum.
Write a separate script that loads a variable Se with $S_e (N)$ from N = 1
to N = 1,000. Show that as N becomes large $S_e$ approaches a limit.
This limit is called Euler\textquotesingle s constant, often represented by the Greek
letter $\gamma$. To 15 digits, Euler\textquotesingle s constant is
$$\gamma =0.577215664901532$$
Add a second output to EulerSum.m that returns the error $\text{|}S_e (N)−\gamma\text{|}$.
Use semilogy to plot this error as a function of N.
\item A square wave can be approximated by a sum of sine waves accordingto
\begin{equation}\label{3.7}
f(x)=\sum_{n=1,3,5,...}^{N}a_n(x)
\end{equation}
where
\begin{equation}\label{3.8}
a_n=\dfrac{4}{n\pi}sin(\dfrac{n\pi x}{L})
\end{equation}
Make an anonymous function that evaluates
$a_n(x)$ and then write a loop that evaluates
f(x) for a given value of N. Use $L= 1$ and plot f(x) from $-5$ to 5. Notice that your function will need to accept two arguments: n and x. Use your code to explore how big
N needs to be to get a good approximation to a square wave. If you get a nice clean
picture of a square wave, make your x grid finer and finer until you
see the Gibbs phenomenon spikes at the points of discontinuity that you learned about in Physics 318.
\end{enumerate}
}
\colchunk{\begin{wrapfigure}{r}{9cm}
\captionsetup{singlelinecheck=off, margin={0cm, 4cm}, justification=raggedright, format=hang}
\caption{  The trajectory for a
home run hit, including the effect of air friction. Note that the
path is not a parabola.}
\label {Figure 3.2}
\includegraphics[width=3cm]{ch33}
\end{wrapfigure}}
\end{parcolumns}
\newpage
\end{document}