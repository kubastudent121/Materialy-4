\documentclass[../main.tex]{subfiles}
\begin{document}
\chapter{Pumping a Swing}
\begin{parcolumns}[colwidths={2=0.25\linewidth}]{2}
\colchunk{
	A playground swing is basically a driven and damped pendulum. But there
are two ways to pump a swing: angular momentum pumping and parametric
oscillation. In this lab we\textquotesingle ll study and numerically model both methods.
\section{Pumping With Angular Momentum}
	You are probably most familiar with angular momentum pumping. In this technique, you sit on the seat and lean back, then lean forward, and lean back,etc. You
	enhance angular momentum pumping when you stretch your legs out in front as
	you swing forward and lean back, then tuck your legs back under as you swing
	backward and lean forward. You can see why this works by imagining yourself
	suspended in outer space with your arms extended to the side. If you were to
	move your right arm up and your left arm down, your body would twist sideways
	in the opposite direction to conserve angular momentum. Now imagine doing
	the same thing with your arms while sitting on a swing. When your body twists
	opposite to your arms to try to conserve angular momentum, friction between
	your jeans and the swing seat will drag the swing with your body and you will
	start the swing moving to the side.\\
	Usually you want to pump a swing forward and backward rather, rather than
	side to side. Since your torso and legs have more mass than your arms, you can
	do a better job of pumping the swing by leaning your body and moving your legs
	than you can by waving your arms. When you twist your body backward, you
	exert a torque on the swing in the forward direction, and when you sit up again
	you create a torque in the other direction. When you repeat these motions at the
	resonant frequency $\omega _0$ of the swing, you will be resonantly driving the pendulum,
	as we discussed in lab 7. This seems to be something that kids on a playground
	just do without knowing any physics at all.\\
	To see how this works analytically, consider a the model of a swinger shown in
	\hyperref[Figure 10.1]{Figure 10.1}\textsuperscript{1} The overall position of the swing is described by $\theta$. The swinger is represented by the masses $m_1$, $m_2$, and $m_3$ and the possibility for the swinger to twist
	is represented by $\phi$. If we make the assumption $m_2\ell _2 = m_3\ell _3$, the Lagrangian for
	this system simplifies to
	\begin{equation}\label{10.1}
		L=\dfrac{1}{2}I_1\dot{\theta}^{2}+\dfrac{1}{2}I_2(\dot{\theta}+\dot{\phi})^{2}+Mq\ell _1 cos\theta
	\end{equation}
	$$ where ~M=m_1 +m_2+m_3,~~I_1=M{\ell _1}^{2},and ~I_2=m_2{\ell _2}^{2}+m_3{\ell _3}^{2}$$
	\subsection*{\small{P10.1}}\label{P10.1}
		 On paper, use the Lagrangian equation of motion for $\theta$, i.e.
		 $$
		 	\dfrac{\partial L}{\partial \theta}-\dfrac{d}{dt}(\dfrac{\partial L}{\partial \dot{\theta}})=0
		 $$
		 to generate the equation of motion for $\theta$. Then assume that the swinger
		twists harmonically, with\\
		\begin{equation}\label{10.2}
		\phi(t)=A+Acos(w_\phi t)
		\end{equation}
		and show that the equation of motion becomes
}
\colchunk{
\begin{wrapfigure}{r}{8cm}
	\captionsetup{singlelinecheck=off, margin={0cm, 4cm}, justification=raggedright, format=hang}
	\caption{A simple model of a
swing being pumped from the
seated position.}
	\label {Figure 10.1}
	\includegraphics[scale=0.8]{ch1010}
\end{wrapfigure} 
}
\end{parcolumns}
\footnotetext{\textsuperscript{1}This model is taken from W. B. Case and M. A. Swanson, “The pumping of a swing from the
seated position”, American Journal of Physics 58, 463-467 (1990).}
\newpage 
\begin{flushright}
\begin{minipage}[t]{10cm}
		\begin{equation}\label{10.3}
			\ddot{\theta}+{\omega _0}^{2}sin\theta = \alpha cos(\omega _\phi t)
		\end{equation}
		where
		\begin{equation}\label{10.4}
			{\omega _0}^{2}=\dfrac{MG\ell _1}{I_1+I_2}~~~~\alpha=\dfrac{I_2A{\omega _\phi}^{2}}{I_1+I_2}
		\end{equation}				
		Compare with \hyperref[7.10]{Eq. (7.10)} and confirm that this is the equation for a driven
	pendulum.
	\subsection*{\small{P10.2}}\label{P10.2}
	 Add some friction to \hyperref[10.3]{Eq. 10.3} by appending a linear damping term $−\gamma \dot{\theta}$ on
	the right$-$hand side, and then solve the modified equation numerically using Matlab. Use the following ballpark numbers for a child on a playground
	swing: $\ell _1 = 2 m, m_1 = m_2 = m_3 = 10 kg, \ell _2 = \ell _3 = 0.5 m, and A = 0.5 rad$
	(about 30 $\textdegree$). By experiment with a backyard swing, we find that $\gamma = 0.1s^{−1} $
	is reasonable. Start from at$-$rest conditions and pump with $\omega _p = \omega _\phi = \omega _0$.
	Plot the solution from $t = 0 ~to ~t = 800$ s and note that it comes to a steady
	state amplitude as driving and damping balance. Then use the following
	code to animate the pumping process.
	\begin{center}
	\textbf{Listing 10.1} (pumpanimate.m)
	\end{center}
\begin{lstlisting}[language=Matlab][numbers=none]
	% Put the code solving the equation above. The code below assumes
	% that you have evenly spaced time steps with the following variables
	% te -> time array
	% xe -> angle theta
	% wp -> the pumping frequency
	% A -> the pumping amplitude
	% l1 -> the main swing length
	% l2 -> the head-to-middle and middle-to-foot length
	tau=te(2)-te(1); L = l1+l2;
	for istep=1:length(te)
	% Position of swing relative to the pivot
	theta = xe(istep);
	xswing=l1*sin(theta);
	yswing=-l1*cos(theta);
	% position of head/legs with respect to swing
	phi=A + A*cos(wp*te(istep));
\end{lstlisting}
\end{minipage}
\end{flushright}
\newpage
\begin{flushleft}
\begin{minipage}[t]{12cm}
\begin{lstlisting}[language=Matlab][numbers=none]
	xpers=l2*sin(phi + theta);
	ypers=l2*cos(phi + theta);
	% Plot the swing and the swinger
	plot([0, xswing],[L,L+yswing],...
	[xswing+xpers,xswing-xpers],[L+yswing-ypers,L+yswing+ypers])
	% Make the x and y dimensions scale equally
	axis([-L/2 L/2 0 L])
	axis square
	% We'd like the plots frames to show at intervals of tau so the movie
	% matches the physical time scale. However, the calculations
	% and plotting take some time, so we decrease the pause a bit.
	% Depending on the speed of your computer, you may need to adjust
	% this offset some.
	pause(tau-0.01)
	end
\end{lstlisting}
\section{Pumping With Parametric Oscillations}
	You can also pump a swing by standing on the seat and doing deep knee bends.
	As you start to swing forward you bend your knees and then stand up hard as
	you go through the bottom of the motion. When you start back you repeat this
	motion by bending your knees and then standing up hard as you go backward
	through the bottom of your motion\textsuperscript{2} This was easy to do on the solid wooden
	seats that swings used to have. However, after a couple of generations of kids
	getting their teeth knocked out by these wicked flying planks, playgrounds put
	in soft flexible seats. These seats are safer, but they are hard to stand on while
	moving your body up and down, so you may not have pumped a swing this way.
	Nevertheless, it is a very good way to pump a swing, although it seems a bit
	mysterious since no rotation is taking place. All you do is move your center of
	mass up and down vertically and rotation of the swing magically appears. The
	name of this mysterious technique is parametric oscillation.
	The so$-$called parametric oscillator equation is\textsuperscript{3}
\begin{equation}\label{10.5}
	\ddot{x}+\gamma \dot{x}+{\omega _0}^{2}(1+\epsilon cos(\omega _p t))x=0
\end{equation}
	Notice that this is different from a driven oscillator because the oscillating term
	cos ($\omega _p t$) is multiplied by x(t). What is happening here is that the natural frequency (one 	of the system parameters) is wiggling in time, which is why this is called a parametric oscillator. In the case of a swing (which is really just a pendulum) you are changing the distance $\ell$ from the top of the chain to your center
	of mass. Since the natural frequency of a pendulum is given by ${\omega _0}^{2}=g/\ell$ as you
	wiggle your center of mass you are wiggling the natural frequency. \hyperref[10.5]{Equation 10.5}
	is simpler than the real equation for a pendulum. We will do the pendulum correctly later in this lab, but for small oscillation angles of the swing \hyperref[10.5]{Eq. 10.5}) gives
	a reasonable approximation to the the motion of a swing.
\end{minipage}
\footnotetext{\textsuperscript{2}There are some nice videos of pumping a swing this way at http://retro.grinnell.edu/academic/
physics/faculty/case/swing.
}
\footnotetext{\textsuperscript{3}L. D. Landau and E. M. Lifshitz Mechanics (Pergamon Press, New York, 1976), p. 80-83, and M.
Abramowitz and I. A. Stegun Handbook of Mathematical Functions (Dover, New York, 1971), Chap.
20.}
\end{flushleft}
\newpage
\begin{parcolumns}[colwidths={2=0.25\linewidth}]{2}
\colchunk{
\subsection*{\small{P10.3}}\label{P10.3}
\begin{enumerate}[label=\alph*]
	\item Use Matlab to solve \hyperref[10.5]{Eq. 10.5} with initial conditions $x(0) = 0$ and
	$v(0) = 1$, and parameters $\omega _0 = 1, \gamma = 0, \epsilon = 0.1, and \omega _p = 1.1$. Plot the
	solution x(t) for a long enough time that you can see that nothing
	much happens except wiggles with some beating between the natural
	motion at $\omega _0$ and the parametric drive at $\omega _p$.\\
	 Pumped swing instability.
	Then run your code again with $\omega _0 = 1, \omega _p = 1, \gamma  = 0, and \epsilon = 0.1$.
	This matches the pumping frequency with the natural frequency of
	the swing, something you might expect would resonantly drive the
	oscillator. Verify that the system is only weakly unstable (meaning that
	the motion slowly grows exponentially with time.) You might have to
	run for a long time to see this instability.
	\item Now run your Matlab code again with $\omega _p = 2$ and watch what happens.
	You should reproduce \hyperref[Figure 10.2]{Figure 10.2}. Verify that $\omega _p = 2\omega _0$ is more unstable
	(i.e. the amplitude grows faster) than $\omega _p = \omega _0$. Once you see the
	$2\omega _0$ instability on your screen, come observe it with the physical
	pendulum at the front of the class.
	\item Show by numerical experimentation that the oscillator is unstable
	at $\omega _p = 2\omega _0$ for all choices of $\epsilon$, but that the instability growth rate is
	small for small $\epsilon$.
	\item Show that when $\omega _p$ is not quite $2\omega -0$ the oscillator is stable for small $\epsilon$,
	but that when $\epsilon$ exceeds some threshold, it becomes unstable again.
	Find this threshold value for $\omega _p = 2.05\omega _0$ and for $\omega _p = 1.95\omega _0$.
	\item ow add damping by setting $\gamma = 0.03$ and show that there is a threshold value of $\epsilon$ even at $\omega _p = 2\omega _0$. Find it by numerical experimentation.
\end{enumerate}
	You should have discovered by now that the best way to parametrically drive
	an oscillator is to use $\omega _p = 2\omega _0$. Is this what you do when you pump a swing by
	standing on the seat? Think about how often you move your center of mass up
	and down in one period of the swing and explain to your TA how $\omega_p$ and $\omega _0$ are
	related as you do this.
\section{Interpreting the Spectrum of the Parametric Oscillator}
	To gain some insight into why $\omega _p = 2\omega _0$ is more unstable than $\omega _p = \omega _p$ 	it is helpful
	to look at the power spectrum of x(t) for the parametric oscillator. In doing this analysis we will use a form of perturbation theory which all physicists love, but
which you may not have seen. So before we look at the spectrum of x(t), let\textquotesingle s do a
perturbation theory problem as a warm$-$up.
Suppose that you wanted to solve the equation
\begin{equation}\label{10.6}
	x^{3}=1+0.1e^{x}
\end{equation}
for a real solution near $x = 1$. This equation is horrible, but if it weren\textquotesingle t for the $e^{x}$
term, it wouldn\textquotesingle t be so bad: $x^{3} = 1$ so$ x = 1$. But look; the exponential term is not
so important since it is multiplied by 0.1, which is small. Shouldn\textquotesingle t we be able to
exploit this smallness somehow? The answer is yes, and here is how to do it, step
by step.
\subsection*{Step 0:} Ignore the small $e^{x}$
term altogether and just solve the easy equation:
\begin{equation}\label{10.7}
	x^{3}=1~~\implies ~~x_0=1
\end{equation}
We call this beginning, and easiest, solution $x_0$ to keep track of which step
we are in.
\subsection*{Step 1:}
 We will now get a better approximation to the solution by writing the
equation down again, but with a twist in the small exponential term. We
will guess that since it is small, it might be OK to replace the horrible $e^{x}$ by
an approximate version of it, namely $e^{x_0}$
\begin{equation}\label{10.8}
x^{3}=1+0.1e^{x_0}~~\implies ~~x^{3}=1+0.1e^{1}~~\implies ~~x_1=(1+0.1e^{1})^{1/3}
\end{equation}
The replacing of $e^{x}$ by $e^{x_0}$ again made the equation easy to solve, which is
good, but we still haven\textquotesingle t found the correct solution.
\subsection*{Step 2:}
 To further improve our solution we repeat step 1, writing
\begin{equation}\label{10.9}
x^{3}=1+0.1e^{x_1}~~\implies ~~x^{3}=1+0.1exp[(1+0.1e^{1})^{1/3}]~~\implies 
\end{equation}
$$x_2=(1+0.1exp[(1+0.1e^{1})^{1/3}])^{1/3}$$
\subsection*{\small{P10.4}}\label{P10.4}
 This procedure starts to look ugly analytically, but if we just want a numerical answer there is no point in writing all of this out. Solve \hyperref[10.6]{Eq. 10.6} by
continuing this step by step approach all the way to 15 significant figures in
the Matlab command window by typing
\begin{lstlisting}[language=Matlab][numbers=none]
format long e
x=1
\end{lstlisting}
and then
\begin{lstlisting}[language=Matlab][numbers=none]
x=(1+0.1*exp(x))^(1/3)
\end{lstlisting}
and then using the $\uparrow$ key to repeat the last step over and over again. Just
watch the result for x and quit when the digits in the answer quit changing.
You should find that the procedure converges to $x = 1.090733645657879$;
verify that this is the solution to the equation
$$
	x^{3}=1+0.1e^{x}
$$
This is a very powerful trick and we will now use it to understand the parametric instability at $\omega _p = 2\omega _0$.
\subsection*{\small{P10.5}}\label{P10.5}
Using $x(0) = 0,~ v(0) = 1, ~\omega _0 = 1,~ \omega _p = 1.3,~ \gamma = 0, ~and ~\epsilon = 0.3$, run your model
from $t = 0~to t~ = 500$ with 214 equally spaced time steps. Take the Fourier
transform and display its power spectrum using a semilogy plot. Our
upcoming analysis will be easier if we consider negative frequencies, so
use ft.m from Chapter 14 of \textit{Introduction to Matlab} and construct your
frequency array appropriately.\\
You should immediately notice the big peaks at $±\omega _0$. This is not a surprise
because what we have is an oscillator at frequency $\omega _0 = 1$ plus a small
perturbation of size $\epsilon$ at frequency $\omega _p = 1.3$. But if you look for a peak at
$\omega = 1.3$, you won\textquotesingle t find it, even though there are plenty of other peaks. Our
job now is to explain why these other peaks are where they are.\\
Since $epsilon$ is small and the damping is weak, let\textquotesingle s begin by ignoring them both
$(\epsilon = 0 ~and~ \gamma = 0)$. (This is step 0 in our perturbation analysis.) Then note that with
these simplifications  is solved by
\begin{equation}\label{10.10}
x_0(t)=Acos(\omega _0t)
\end{equation}
Now we will proceed by perturbation theory as we did in the previous problem,
like this. Make a more precise guess at the solution by writing \hyperref[10.5]{Eq. 10.5} down
again, but with $x _0$ in place of x in the small term ${\omega _0}^{2}\epsilon cos(\omega _p t)x:$
\begin{equation}\label{10.11}
\ddot{x}+\gamma \dot{x} +{\omega _0}^{2}(1+\epsilon cos(\omega _p t))x _0=0~~\implies~~\ddot{x}+\gamma \dot{x} {+\omega _0}^{2}x \approx-\epsilon {\omega _0}^{2}cos(\omega _p t))x _0
\end{equation}
With $x_0 = Acos (\omega _0 t)$ this is just a complicated version of the driven harmonic
oscillator.
\subsection*{\small{P10.6}}\label{P10.6}
Use Mathematica to solve this equation with $\gamma = 0$. You may need to recall
that inhomogeneous linear differential equations like this have solutions
of the form $x = x_h + x_p$, where $x_h$ is the homogeneous solution (without
the parametric driving term) and where $x_p$ is the particular solution with
the driving term included. Mathematica will give you a pretty complicated
answer, but if you look at it closely you will see that the homogeneous
solution is just our friend $x_0 = Acos (w_0 t)$ and that the particular solution
(the messy part) can be written as a sum of terms that involve sines and cosines at new frequencies. It\textquotesingle s as if the driving force had a split personality
involving more than one driving frequency. This is exactly right, as you can
see by using the identity
}
\colchunk{
	\begin{wrapfigure}{r}{7cm}
	\captionsetup{singlelinecheck=off, margin={3cm, 0cm}, justification=raggedright, format=hang}
	\caption{A simple model of a
swing being pumped from the
seated position.}
	\label {Figure 10.2}
	\includegraphics[scale=0.8]{ch9zd11}
\end{wrapfigure}
}
\end{parcolumns}
\begin{flushleft}
\begin{minipage}[t]{12cm}
$$
	cos\alpha cos\beta = \dfrac{1}{2}(cos(\alpha +\beta) + cos(\alpha + \beta))
$$
to rewrite the driving term on the right side of the differential equation
above. Do the math and see what frequencies turn up. Can you see
these \textquotesingle\textquotesingle sideband\textquotesingle\textquotesingle frequencies in your spectrum from \hyperref[P10.5]{P10.5} and in your
Mathematica solution?\\
In this first$-$order perturbation theory we see that in addition to the peak at $\omega _0$,
there are two other, smaller, sideband contributions at $(\omega _0 +\omega _p)$ and at $(\omega _0 −\omega _p)$.
If we now take the second step in perturbation theory the driving term will be
\begin{equation}\label{10.12}
-\epsilon \omega _0 cos(\omega _p t))x _1
\end{equation}
If you use the trig identity above for this second step in the perturbation theory,
you will find that each of the frequency components from the first$-$order step is
multiplied by cosωp t and that they then produce new sidebands shifted again
from the first$-$order frequencies $\pm \omega _p$. These second$-$order sidebands are smaller
in magnitude than the first$-$order sidebands because in each step the new driving
term is multipled by another factor of $\epsilon$. But they are clearly there in the spectrum.
This procedure, of course, never ends, so it is easy to see that this equation can
produce a very rich spectrum.\\
Now, what does this have to do with the observed instability at $\omega _p = 2\omega _0$? Well,
as we saw in first$-$order perturbation theory, when we parametrically oscillate at
ωp the system looks like a driven oscillator with driving frequencies at $\omega _0 \pm \omega _p$.
When $\omega _p = 2\omega _0$, the sum and difference frequencies fall at $3\omega _0$ and $−\omega _0$. Since
one of the apparent driving frequencies is at $\omega _0$ (note that $−\omega _0$ is just as resonant
as $\omega _0$), the system feeds back on itself and is unstable.\\
This effect also allows us to see why $\omega _p = \omega _0$ is not the most unstable choice.
The reason is that with this choice the sideband frequencies are 0 and $2\omega _0$, neither
one of which is resonant. But the second$-$order sidebands are $−\omega _0, \omega _0, \omega _0$, and
$3\omega -0$. Three out of four come back to $\omega _0$ in second order, so there is a possibility of
instability due to this nonlinear resonance. But because it takes two perturbation
steps to get to this resonance, and since each step involves another power of $\epsilon$,
this choice for $\omega _p$ is less unstable.
\subsection*{\small{P10.7}}\label{P10.7}
 Explain all of the frequency peaks in your spectrum from \hyperref[P10.5]{P10.5} using the
concepts explained above. Explain what the amplitudes of the various
peaks mean and how the $\omega _p = 2\omega _0$ instability arises.
\section{Parametrically unstable pendulum}
When you pump a real swing, your oscillation amplitude doesn\textquotesingle t become infinite
because the swing is a pendulum, not a harmonic oscillator. Using the Lagrangian formulation of mechanics to obtain the equation of motion of a pendulum whose
length $\ell (t)$ is changing with time, and adding some damping because of air
friction, gives us the equation
\begin{equation}\label{10.13}
\ddot{\theta}+2\dfrac{\dot{\ell}}{\ell}\dot{\theta}+\gamma \dot{\theta}+\dfrac{g}{\ell}sin\theta=0
\end{equation}
If we let the length change sinusoidally at frequency $\omega _p$ by only the small amount
$\Delta L$ about the constant length $L_0$, then

\begin{equation}\label{10.14}
\ell (t)=L _0 +\Delta L cos \omega _p t
\end{equation}
\end{minipage}
\end{flushleft}
\begin{flushright}
\begin{minipage}[t]{12cm}
\subsection*{\small{P10.8}}\label{P10.8}
Use Matlab to solve \hyperref[10.13]{Eq. 10.13} with (\hyperref[10.14]{10.14}) using the following realistic
parameter values. A typical backyard swing has a length of about $L_0 = 2 m$.
As you do deep knee bends you move most of your mass up and down, so
the $\Delta L$ of your parametric oscillation is about half the distance you drop
your body during the bend. Use $\gamma = 0.1 s^{−1}$
again for the decay. In your
numerical solution gradually increase $\Delta L$ from zero up to around 0.3 m with
$\omega _p = 2\omega _0$ and find the threshhold value of $\Delta L$ at which the swing becomes
unstable. (Explain to your TA why the pendulum amplitude doesn\textquotesingle t just
keep getting bigger forever.)
\end{minipage}
\end{flushright}
\end{document}