\documentclass[../main.tex]{subfiles}
\begin{document}
\begin{flushleft}
\begin{minipage}[t]{12cm}
\chapter{Introduction to Matlab}
		Differential equations are the language of physics, but most of the interesting
	problems involve differential equations that can\textquotesingle t be solved analytically. In this
	course we \textquotesingle ll learn techniques to numerically solve differential equations with the
	goal of studying interesting physics problems. Our first step is to learn the basics
	of the Matlab programming language.\\
\section{Basic Syntax}
\subsection*{\small{P1.1}}\label{P1.1}
 	Read and work through \textit{Introduction to Matlab}, Chapter 1. Type and execute
	all of the material written in this kind of font. After you have worked
	through the chapter, use the Matlab command line to define the matrices
\begin{lstlisting}[language=Matlab][numbers=none]
	A=[1,2,3;4,5,6;7,8,9]
	B=[1,4,5;9,6,3;2,3,1]
\end{lstlisting}
	Also define the row and column vectors
\begin{lstlisting}[language=Matlab][numbers=none]
	v1=[1,1,2]
	v2=[0.40824829 ; -0.81649658 ; 0.40824829]
\end{lstlisting}
\begin{enumerate}[label=\alph*]
	\item Use both * and .* to multiply A and B. Explain the difference.
	\item Perform the operation A./B and explain the result.
	\item Perform the operations $A*v_1$, $v_1*A$, and $A*v_2$ and explain the results.
	\item Multiply only the center elements of A and B together
	\item Perform the operation exp$(A+i*B)$ and explain what it means.
	\item Extract the center column of A, and then divide each element of this
	column by the corresponding element in $v_2$.
\end{enumerate}
\section{Writing Scripts}
\subsection*{\small{P1.2}}\label{P1.2}
	 Read and work through \textit{Introduction to Matlab}, Chapter 2. Type and execute
	all of the material written in this kind of font. Then complete the
	following exercises:
\end{minipage}
\end{flushleft}
\newpage
\begin{flushright}
	\begin{minipage}[t]{12cm}
		\begin{enumerate}[label=\alph*]
 \item In your Freshman physics course, you learned that in the absence of
	air resistance, a battleship\textquotesingle s projectile travels a horizontal distance
	$$d = \dfrac{v^{2}}{g}sin(2\theta)$$
	where v is its initial speed and $\theta$ is the initial angle above the horizontal. Write a Matlab script that asks the user to enter a value for v in m/s
	and $\theta$ in degrees, and then calculates and prints the range formatted
	with one decimal place, like this:\textquotesingle\textquotesingle Range: 45.2 meters\textquotesingle\textquotesingle. Remember
	that the standard trig functions are permanently set to use radians,
	but degree versions also exist. Use your program to find the proper
	angle to hit a target exactly 10 km away if the initial velocity is 750 m/s.
	A battleship\textquotesingle s guns can\textquotesingle t elevate above 45 degrees.
\item  A planet\textquotesingle s velocity with respect to its star is $v_1 = 30000 \vec{x}~ m/s$ when it is
	hit by an asteroid with velocity 
	$$v2 = (−5000 \vec{x} +8000 \vec{y} +1000 \vec{z}) m/s$$
	The planet has mass $m_1 = 6*10^{24} kg$ and the asteroid has mass $m2 =
	1 * 10^{19} kg$. Write a script that defines the masses and velocities of
	the planet and asteroid using the variables m1, m2, v1, and v2. Then
	calculate and display the final velocity of the planet after the collision:
	$$v_f = \dfrac{m_1*v_1+m_2*v_2}{m_1+m_2}$$
	Your variables and your answer should be vectors. If it bothers you
	that the planet\textquotesingle s velocity didn\textquotesingle t change, think about the precision you
	are using to display it.
\end{enumerate}
\section{Loops, Logic, and Debugging}
\subsection*{\small{P1.3}}\label{P1.3}
	Read and work through \textit{Introduction to Matlab}, Chapters 3$-$4. Type and
	execute all of the material written in this kind of font, and run the
	code in the listings.
\begin{enumerate}[label=\alph*]
\item Write a script that uses a for loop to find the factors of 24 by testing
	every number from 1 to 12 using the mod function. Once you are sure
	your code correctly finds all of the factors of 24, use it to find all of
	the factors of 18648. To pass this exercise off, you must have at least 4
	comment lines in your code.
\item Use a while loop to automate the process of finding the answer to the
	battleship range problem in \hyperref[P1.2]{P1.2}(a). Start the elevation at zero degrees
	and increase it in steps of 0.1 degrees until your range exceeds 5 km,
	and then break out of the while loop.
\item Your roommate has borrowed \${}100,000 in student loans. The loan
	charges a 6\%{} annual interest rate, and interest charges are applied
	each month. In other words, every month the loan charges a 0.5\%{}
	interest fee on the remaining balance. This means that the amount
	by which you reduce your balance each month is not the amount you
	pay, but your payment minus the monthly interest. Your roommate
	plans to pay your loan off by paying \${}1,000 per month. Write a program using loops and logic to calculate how long it will take to finish
	paying off the loan. Don\textquotesingle t get fancy and derive an analytic formula for
	compound interest payoff time. Just write a loop to track the balance
	over time, and break out when the balance gets to zero.
\end{enumerate}
\end{minipage}
\end{flushright}
\end{document}
