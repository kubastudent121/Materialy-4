\documentclass[../main.tex]{subfiles}
\begin{document}
\vfill
\begin{parcolumns}[colwidths={2=0.3\textwidth}]{2}
\colchunk{
\chapter{Calculus and a Bouncing Ball}
\begin{flushleft}
For the past couple of labs we\textquotesingle ve focused on ways to visualize the solutions to
differential equations using phase$-$space plots and qualitative analysis. In this lab
we\textquotesingle ll begin to learn how to solve differential equations numerically. To do this, we
represent functions of space and time using discrete grids rather than continuous
variables. Then we approximate derivatives as finite differences on this discrete
grid rather than the infinitesimally small differences in analytic calculus. We
begin by exploring how to do calculus on grids. Then we\textquotesingle ll use these ideas to
develop a crude technique for numerically solving the differential equations for a
bouncing ball.
\section{Calculus on a Discrete Grid}
\subsection*{\small{P4.1}}\label{P4.1}
 Read and work through \textit{Introduction to Matlab}, Chapter 8. Type and execute
all of the material written in this kind of font. Then complete the
following exercises.
\end{flushleft}
\begin{enumerate}[label=\alph*]
\item Use the simple mid$-$point rule to numerically do the integral
\begin{equation}\label{4.1}
\int_{0}^{2} x^2e^{-x}cosx \,dx
\end{equation}
Experiment with different values of N until you are confident that you
have the answer correct to 6 decimal places. Then verify that you did
it right by doing the same integral using Matlab\textquotesingle s integral command
with an anonymous function.
\item Consider the function $f(x)=e^{x}$.The derivative of this function is obviously $f'(x)=e^{x}$, but imagine you didn\textquotesingle know that, and numerically evaluate $f'(x)~at~x=1$ using both the forward and centered difference
approximation to the first derivative with a step size h = 0.5. Then
compare the approximations with the analytic answer $(i.e.f'(1)=e).$ When you are sure you have the numerical derivatives coded correctly,
switch h to be an array with values
$$[\dfrac{1}{2},\dfrac{1}{4},\dfrac{1}{8},...,\dfrac{1}{2^{65}}]$$
Then write a for loop that calculates the error of the two derivative formulas as 
\begin{lstlisting}[language=Matlab][numbers=none]
err=abs(fp/exp(1)-1)
\end{lstlisting}
(where fp is the numerical derivative) for each value of h, and stores the error in another array that
is the same size as h. Finally, make overlaid loglog plots of the two
errors vs. h. Show that the centered difference formula works better,
but that both formulas are bad for very small values of h.
Explain why very small values of h make the approximate derivative
be wrong, giving zero instead of a good approximation to f\textquotesingle .The
section below on roundoff will be helpful.
\end{enumerate}
}
\colchunk{}
\end{parcolumns}
\begin{flushright}
\begin{minipage}[t]{10cm}
\section{Roundoff}
The effect illustrated in exercise 4.1(b) is called roundoff and it rears its ugly head
every time you subtract two numbers on a computer. To understand roundoff, consider the following two 15-digit numbers: $a = 1.2345678912345$ and
$b = 1.2345678918977$. These are impressively accurate numbers, but their difference is not so impressive: $b−a = .0000000006632$. Where did all of the significant
digits go; we started with 15 and now we only have 4? The problem is that the
numbers were so close together that subtraction made most of the significant
figures go away. When you work with numerical data on a computer you only
have a finite number of significant digits (15 in Matlab), so you have to be careful
when you subtract. And because subtraction is the key idea in differentiating, we
have to be careful about how we choose our step size h. As you can see in this
exercise, making it very small makes things worse, not better.
\section{Numerical Solutions to Differential Equations}
Now that we understand the basics of taking derivatives on a grid, let\textquotesingle s look at how
to numerically solve differential equations. Consider the motion of a projectile
near the surface of the earth with no air resistance. The differential equations
that describe the projectile are
\begin{equation}\label{4.2}
\dfrac{d_x}{d_t}=v_x~~~~\dfrac{d_y}{d_t}=v_y\\
\dfrac{dv_x}{d_t}=0~~~~\dfrac{dv_y}{d_t}=-g
\end{equation}
along with some initial conditions, x(0), y(0), $v_x (0)$, and $v_y (0)$. This set of equations is easily solved analytically, but imagine that we did\textquotesingle t have an analytic
solution. How could we numerically model the motion of the projectile?
The basic idea behind a numerical solution is to think of your independent
variable (time in this case) as being a discrete grid rather than a continuous
quantity. It is easiest to represent time with an evenly spaced grid $[t_0,t_1,t_2,...]$
with $t_0 = 0, t_1 = \tau , t_2 = 2\tau ,$ etc. Then we label the dependent variables (space in
this case) using the same indexing as the time grid, like this: 
$$
x0 \equiv x(0), x1 \equiv x(\tau ),
x2 \equiv x(2\tau ),
$$
etc. With this notation, we can write the equations in (\hyperref[4.2]{ Eq. 4.2}) using the
\end{minipage}
\end{flushright}
\newpage
\begin{parcolumns}[colwidths={2=0.3\textwidth}]{2}
\colchunk{
(inaccurate) forward difference approximation of the derivative that you learned
about in the reading:
\begin{equation}\label{4.3}
\dfrac{x_{n+1} -x_n}{\tau }=v_{x,n}~~~~\dfrac{y_{n+1} -y_n}{\tau }=v_{y,n}~~~~
\dfrac{v_{x,n+1} -v_{x,n}}{\tau }=0~~~~\dfrac{v_{y,n+1} -v_{y,n}}{\tau }=-g
\end{equation}
Notice that the left sides of these equations are centered on the time $t_n+1/2$, but
the right sides are centered at time $t_n$. This makes this approach inaccurate, but
if we make $\tau$ small enough it can work well enough to see the principles involved.
By solving the equations in (\hyperref[4.3]{ Eq. 4.3}) we can obtain a simple algorithm for stepping
our solution forward in time:
\begin{equation}\label{4.4}
x_{n+1}=x_n+v_{x,n^\tau}~~~~y_{n+1}=y_n+v_{y,n^\tau}~~~~
v_{x,n+1}=v_{x,n}~~~~v_{y,n+1}=v_{y,n}-g \tau
\end{equation}\\
This method of approximating solutions is called Euler\textquotesingle s method. In general, it\textquotesingle s
not very good, especially over many time steps. However, it provides a foundation
for learning other better methods.
\subsection*{\small{P4.2}}\label{P4.2}
 Make a program in Matlab to model the motion of a ball bouncing on the
floor using Euler\textquotesingle s method. In your script, define the initial position of the
ball with $x=0 ~and~ y=1$, and the initial velocity with $v_x=1~ and~ v_y=0$. Then
write a while loop to step the position and velocity forward in time using
 \hyperref[4.4]{ Eq. 4.4}. Have your while loop exit when $x > 10$. Use new variable names
for the quantities at time level $n +1$, like this:
\begin{lstlisting}[language=Matlab][numbers=none]
xnew = x + vx*tau;
vynew= vy - g*tau;
etc.
\end{lstlisting}
Then when you have advanced all four quantities, update the current values
to get ready for the next step, like this:
\begin{lstlisting}[language=Matlab][numbers=none]
x=xnew;
y=ynew;
etc.
\end{lstlisting}
\begin{enumerate}[label=\alph*]
\item To simulate bouncing, put an if statement in your loop that checks if
y is less than zero. When it is, make $v_y$ positive like this
\begin{lstlisting}[language=Matlab][numbers=none]
vy=abs(vy)
\end{lstlisting}
Make a movie by plotting the position of the ball as a dot each time
the loop iterates, like this:
\begin{lstlisting}[language=Matlab][numbers=none]
plot(x,y,'.')
axis([0 10 0 1.5])
pause(0.001)
\end{lstlisting}
\item Our bouncing condition in part (a) is lousy. Make it better by adding
some more logic that does the following:
(i) Test to see if y will go less than zero on this time step, but don\textquotesingle t
actually change y yet.
(ii) If y won\textquotesingle t go less than zero this step, just do a regular Euler step.
(iii) If it will go negative this time step, determine a smaller time step
$\tau_1$ such that an Euler step will take the ball to $y = 0$. Then take an
Euler step with $\tau_1$. After taking this small step, make the y$-$velocity
positive as before
\begin{lstlisting}[language=Matlab][numbers=none]
vy=abs(vy)
\end{lstlisting}
and then take an Euler step of $\tau_2 = \tau~-~ \tau_1$ to finish off the time interval.
Play with different values of $\tau$ and notice that even with this improved
bouncing condition, Eule\textquotesingle s method is always unstable (i.e. the amplitude of the bounce continues to grow). This is a limitation of Euler\textquotesingle s
method, and we\textquotesingle ll develop better methods to overcome this shortcoming next time.
\item  Make your model look more realistic by adding some energy loss
during the bounce process by changing your bounce code to look like
this
\begin{lstlisting}[language=Matlab][numbers=none]
vy=0.95*abs(vy)
\end{lstlisting}
This damping will mask the growth of Euler\textquotesingle s method for a suitably
small $\tau$.
\end{enumerate}
}
\colchunk{
\begin{wraptable}{l}{4cm}
	\begin{tabular}{m{4cm}}
The name Euler does not
rhyme with \textquotesingle\textquotesingle cooler\textquotesingle\textquotesingle; it
rhymes with \textquotesingle\textquotesingle boiler\textquotesingle\textquotesingle . You
will impress your fellow students and your professors
if you give this important
name from the history of
mathematics its proper pronunciation
\end{tabular}
\end{wraptable}
}
\end{parcolumns}
\end{document}
