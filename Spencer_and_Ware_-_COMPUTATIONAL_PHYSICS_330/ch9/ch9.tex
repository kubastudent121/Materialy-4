\documentclass[../main.tex]{subfiles}
\begin{document}
\newpage
\begin{parcolumns}[colwidths={2=0.30\linewidth}]{2}
\colchunk{
\chapter{Fourier Transforms}
	\section{Fourier Transforms}
Suppose that you went to a Junior High band concert with a digital recorder and
made a recording of Mary Had a Little Lamb. Your ear told you that there were a
whole lot of different frequencies all piled on top of each other, but perhaps you
would like to know exactly what they were. You could display the signal on an
oscilloscope, but all you would see is a bunch of wiggles. What you really want is
the spectrum: a plot of sound amplitude vs. frequency.
The mathematical method for finding the spectrum of a signal f (t) is the
Fourier transform
\begin{equation}\label{9.1}
g(\omega )=\dfrac{1}{\sqrt{2\pi}}\int_{-\infty}^{\infty}f(t)e^{i\omega t}  \,dt
\end{equation}
If you remember Euler\textquotesingle s relation e
$i\omega t = cos(\omega t)+i sin(\omega t)$, you can see that the
real part of $g (\omega )$ is the overlap of your signal with $cos(\omega t)$ and the imaginary
part of $g (\omega )$ is the overlap with sin$(\omega t)$\textsuperscript{1} Often, we aren\textquotesingle t interested in the phase
information provided by the complex nature of g $(\omega )$, so we just look at the power
spectrum $P(\omega )$
\begin{equation}\label{9.2}
P(\omega ) = {|g(\omega )|}^{2}
\end{equation}
$P(\omega )$ gives the signal intensity as a function of frequency without any phase
information. In this lab, we will learn how to make these types of plots.
\section{FFTs and Fourier Transforms}
\subsection*{\small{P9.1}}\label{P9.1}
 Read and work through \textit{Introduction to Matlab} Chapter 13. Type and
execute all of the material written in this kind of font.
\subsection*{\small{P9.2}}\label{P9.2}
 On the class web site is a file called \textquotesingle\textquotesingle Beethoven.wav\textquotesingle\textquotesingle that has the first four
notes of Beethoven\textquotesingle s 5th symphony. Save it to your computer and listen to
it. Load the sound waveform into the matrix f using\\
\begin{lstlisting}[language=Matlab][numbers = none]
f = audioread('beethoven.wav');
\end{lstlisting}
Then construct the corresponding t time series by noting that the recording
was sampled at 44100 points/second. Plot the signal versus time and plot
its power spectrum versus $\nu$ (not $\omega$) over the range 0$-$1000 Hz with both a
linear scale and with semilogy. (You should get in the habit of looking at

}
\colchunk{
\begin{wrapfigure}{r}{9cm}
	\captionsetup{singlelinecheck=off, margin={1cm, 4cm}, justification=raggedright, format=hang}
	\caption{The power spectrum of
the first four notes of Beethoven\textquotesingle s
5th symphony.}
	\label {Figure 9.1}
	\includegraphics[scale=0.8]{ch999}
\end{wrapfigure}
}
\end{parcolumns}
\footnotetext{\textsuperscript{1}People often work with complex signals, in which case this separation is less clear.}
\newpage
\begin{parcolumns}[colwidths={2=0.3\linewidth}]{2}
\colchunk{
	spectra with a log scale to see structure that may not be evident on a linear
scale.)\\
Now we need to make sense of the spectrum. The short notes at the beginning of the music are the note \textquotesingle\textquotesingle G\textquotesingle\textquotesingle (repeated three times) played in octaves
by the violins/violas (394 Hz), cellos (197 Hz), and basses (99 Hz). The last
note is an \textquotesingle\textquotesingle E$-$flat,\textquotesingle\textquotesingle again played in octaves by the various stringed instruments (312 Hz, 156 Hz, and 78 Hz). Identify each of these peaks on the
spectrum, and explain what their relative amplitudes mean.\\
Note that there are also smaller peaks at 234 Hz, 468 Hz, 624 Hz, 788 Hz,
and 936 Hz. Explain where these extra peaks come from, and how each of
the smaller peaks are connected to the notes in the four$-$note theme (see
\hyperref[Figure 9.2]{Fig. 9.2}).\\
To convince yourself that you know what you are doing, split the time series
into two pieces: one that contains the first three short notes, and a second
that only contains the one last long note. Then repeat the analysis above
and compare your two new spectra to the original one. Show the TA all
three spectra and explain the origin of all of the peaks.
\section{The Uncertainty Principle}
The uncertainty principle connects the duration of a signal in time with the
spread of its spectrum. It was made famous in quantum mechanics by Werner
Heisenberg, but it is really an idea from classical wave physics\textsuperscript{2} which we can
understand by using the fft.\\
Suppose that we have a time signal which has a frequency $\omega _0$, but which only
lasts for a finite time $\Delta t$. For example, consider the Gaussian function
\begin{equation}\label{9.3}
	f(t)=cos(\omega _0 t)e^{-{(t-t_0)}^{2}/W^{2}}
\end{equation}
which has a \textquotesingle\textquotesingle bump\textquotesingle\textquotesingle centered at $t_0$ with a width controlled by W . Because the
signal oscillates at $\omega _0$ we would expect to see a peak in the spectrum at $\omega = \omega _0$.
This frequency peak also has a well$-$defined width, and this width is related to the
width of the signal in time through the uncertainty principle.
\subsection*{\small{P9.3}}\label{P9.3}
 Write a Matlab script to build f (t) from \hyperref[9.3]{Eq. 9.3}, with $t_0$ chosen so that
the bump is in the center of your time window. Plot f (t) and its power
spectrum for $\omega _0 = 200s^{−1}$
and $W = 10$, 1, and 0.1.\\
Choose appropriate values for your number of points N and your time step\\
$\tau$ so that\\
(i) fft will run fast
(ii) you can see frequencies up to$ \omega = 400 s^{−1}$ without aliasing trouble
(iii) your spectral resolution will be at least $d\omega = 0.2 s^{−1}$
To see where the uncertainty principle is lurking in these plots, visually
measure and write down the full width at half maximum (FWHM) of the
time signal $(\Delta t)$ and FWHM of the frequency peak $(\Delta \omega plot)$. Write these
measurements in \hyperref[Table 9.1]{Table 9.1} for each value of W . Then deduce a rough
product relation $\Delta \omega \Delta t$ $\approx$ const between the width of the time signal and
the width of the frequency peak from this data\textsuperscript{3} \\
You have probably experienced the uncertainty principle when listening to
music be narrow relative to the location of the peak. So for a flute playing a high note at $\omega = 6000s^{-1}$ to produce a spectru, with, say a $1%$width requires$\Delta \omega
=(0.01)(6000s^{-1})=60s^{-1}$.Then the uncertainty principle tells us that this
note can be produced by only holding it for the relatively short time of
\begin{equation}
	\Delta t \approx \dfrac{1}{60} = 0.017 s
\end{equation}
where we have arbitrarily chosen $\Delta \omega \Delta t = 1$ to make the calculation. But when
a tuba plays a low note around $\omega  = 200 s^{−1}$
, the same calculation using $\Delta \omega  =
(200)(.01) = 2$ gives a note$-$duration of only
\begin{equation}
	\Delta t \approx \dfrac{1\pi}{\Delta \omega} \approx \dfrac{1}{2} = 0.5s
\end{equation}
}
\colchunk
{
\begin{wrapfigure}{l}{6cm}
\captionsetup{singlelinecheck=off, margin={2cm, 0cm}, justification=raggedright, format=hang}
\caption{ \small{A string\textquotesingle s fundamental mode of vibration has nodes
at the ends and an antinode in
the middle. However, the string
can also vibrate in harmonic
modes with nodes between the
ends. When a musician drags
her bow across a string, she
excites mostly the fundamental, but the harmonics are also
present. The frequencies of these
modes are: $\nu _0$ = fundamental,
$2\nu _0$ = second harmonic, $3\nu _0$ =
third harmonic, etc.}}
\label {Figure 9.2}
\includegraphics[width=4cm]{ch992}
\end{wrapfigure} 
}
\end{parcolumns}
\footnotetext{\textsuperscript{2}The weirdness of quantum comes not from the fact that waves obey the uncertainty principle,
but from the idea that things like electrons behave like waves.}
\newpage
\begin{parcolumns}[colwidths={2=0.3\linewidth}]{2}
\colchunk{
Now tubas can play faster than this, but if you listen carefully, when they do
their sound becomes \textquotesingle\textquotesingle \textquotesingle\textquotesingle, which simply means that the note isn\textquotesingle t a very pure
frequency, corresponding to a wide frequency peak\textsuperscript{4} Your ear$/$brain system also
helps you out here. It is pretty talented at turning lousy signals into music, so you
can still enjoy \textquotesingle\textquotesingle Flight of the Bumblebee\textquotesingle\textquotesingle even when played by a tuba\textsuperscript{5}.\\
You can also hear this effect simply by clapping your hands. If you cup your
hands when you clap, you trap a lot of air, which responds rather slowly to your
clap. This makes a larger value of $\Delta t$, which in turn means that $\Delta \omega$ is smaller,
corresponding to the low frequencies that make up the low, hollow boom of a
cupped clap. But if you slap your third and fourth fingers quickly on your palm
you trap almost no air, resulting in a very small $\Delta t$, and hence, via the uncertainty
principle, a larger $\Delta \omega $. And a larger $\Delta \omega$ means a higher set of frequencies in the
sound of your clap, which you can clearly hear as a higher$-$pitched burst of sound.
\section{Windowing}
Review the material on windowing in \textit{Introduction to Matlab}, then work through
the following problem.
\subsection*{\small{P9.4}}\label{P9.4}
 Modify Listing 13.1 in \textit{Introduction to Matlab} so that it uses the following
time signal
\begin{lstlisting}[language=Matlab][numbers=none]
	f=sin(t)+.5*sin(3*t)+.4*sin(3.01*t)+.7*sin(4*t)+.2*sin(6*t);
\end{lstlisting}
Plot the power spectrum versus ω and verify the relative amplitude problem
discussed in the windowing section in \textit{Introduction to Matlab}. To make the
ratio issue clear, normalize the spectrum so the biggest peak has height 1
(i.e. plot P/max(P) instead of P).
Multiply the time signal by a Gaussian window function like this
\begin{lstlisting}[language=Matlab][numbers=none]
win = window(@gausswin,length(f),alpha)';
f = f .* win;
\end{lstlisting}
he transpose operator (\textquotesingle ) at the end of the first line switches the window
from a column vector to a row vector so that the multiplication works. The
parameter alpha is specific to a Gaussian window, and is related to \hyperref[9.3]{Eq. 9.3}
via $\alpha \infty 1/W —i.e$. a bigger α creates a narrower signal in time. Try several
values of alpha and look at plots of win and f.*win to see what the window
function does.\\
Make the window really narrow with $alpha=25$ and plot the power spectrum of f.*win. Look at the peaks at $\omega = 1,4,6$, and verify that the relative
amplitudes are now right on. (Remember that power is proportional to
amplitude squared.) But what happened to the peaks at $\omega = 3~ and~ \omega = 3.01$?
We\textquotesingle ve made the peaks so broad that they\textquotesingle ve smooshed into each other due
to leakage. Find an alpha that is a good compromise between getting the
right peak amplitude and maintaining good resolution. Explain the concepts of windowing and leakage, and tell how they relate to resolving the
height and width of closely spaced peaks.
\subsection*{\small{P9.5}}\label{P9.5}
Use Matlab to numerically verify the trig identity $cos^{4}(t)=3/8+(1/2)
cos(2t)+(1/8)cos(4t)$ by plotting the Fourier transform of the function. You will need
to choose an appropriate time series and window function to see the relationships accurately.
}
\colchunk{
\begin{wrapfigure}{r}{9cm}
\captionsetup{singlelinecheck=off, margin={0cm, 4cm}, justification=raggedright, format=hang}
\caption{Enter your data here}
\label {Table 9.1 }
\includegraphics[width=4cm]{ch93}
\end{wrapfigure}
}
\end{parcolumns}
\footnotetext{\textsuperscript{3}This is not a mathematically rigorous uncertainty relation, but it illustrates the idea.}
\footnotetext{\textsuperscript{4}The length of the tuba also contributes to the \textquotesingle\textquotesingle muddyness\textquotesingle\textquotesingle of the sound, since it takes a while
for sound to propagate back and forth between the mouthpiece and the bell and set up the standing
wave. This causes a messy \textquotesingle\textquotesingle attack\textquotesingle\textquotesingle transient at the beginning of each note, which means you have
less of the sustained pitch to listen to.}
\footnotetext{\textsuperscript{5}At tuba frequencies, your ear/brain system can perceive pitch for pulses containing only a few
cycles.}
\newpage 

\end{document}