\documentclass[../main.tex]{subfiles}
\begin{document}
\begin{parcolumns}[colwidths={2=0.3\linewidth}]{2}
\chapter{Chaos}
\colchunk{
\section{The van der Pol Oscillator}
Consider the following non-linear oscillator equation, called the van der Pol
oscillator:\textsuperscript{1}
\begin{equation}\label{12.1}
\ddot{x}-\epsilon (\ell ^{2}-x^{2})\dot{x}+{\omega _0}^{2}x=0
\end{equation}
This is a simple model differential equation for systems that have an external
source of energy which causes the resting state $(x = 0, v = 0)$ to be unstable,
but which also have sufficient damping that the instability cannot grow to an
arbitrarily large amplitude.\\
Begin by studying \hyperref[12.1]{Eq. 12.1} and convincing yourself that the resting state is
indeed unstable, but that large amplitude motion is damped (on average). You
can\textquotesingle t see that $x = 0, v = 0$ is unstable by starting the system there and waiting for
something to happen, because nothing will happen. This is an equilibrium point
and if you start it there it will remain there forever. To test for stability, start the
system in a point very close to equilibrium and watch to see if it stays near the
equilibrium point, or runs away from it. Appropriate initial conditions to test
for stability might be $x = 0.0001, v = 0$. The phase$-$space flow plot, made with
quiver, in \hyperref[Figure 12.1]{Figure 12.1} illustrates these two features. Notice the arrows leading away
from the origin and the general inward flow at the outer edges of the picture. The
flow is not uniformly inward, however, and later in this lab you will see the effect
of the squeezed inward flow patterns visible in the figure.
\subsection*{\small{P12.1}}\label{P12.1}
\begin{enumerate}[label=\alph*]
\item Use Matlab\textquotesingle s ode45 to solve \hyperref[12.1]{Eq. 12.1} numerically for $\epsilon = 0.3, \omega _ 0 =
1.3, and \ell = 1$. Use options=odeset(\textquotesingle RelTol\textquotesingle ,1e$-$5) to set the
accuracy of ode45 at a level that will make it possible to do long runs
in a reasonable time. (We would normally use a smaller tolerance
than this, but we only have 3 hours together). Make both a plot of x vs.
t as well as a phase space plot of v vs. x for a bunch of different initial
conditions. Notice that the phase space plot eventually settles on the
same curve for any initial conditions you pick. The phase space curve
on which the solutions settle is called a limit cycle
\item  Repeat part (a) for $\epsilon = 1~ and~ \epsilon = 20$ and note how the limit cycle
changes shape. Also plot the power spectrum of x(t) with semilogy
using an axis command to display the spectrum from $\omega  = 0 ~to~ \omega = 20$
and note where the major peaks are. Remember to interpolate x(t)
onto an even time grid before computing the Fourier transform.
\end{enumerate}
}
\colchunk{
\begin{wrapfigure}{r}{8cm}
	\captionsetup{singlelinecheck=off, margin={0cm, 4cm}, justification=raggedright, format=hang}
	\caption{ Flow in phase space
for the Van der Pol oscillator with $\omega _0=1, \ell =1,~ and~ \epsilon = 1$}
	\label {Figure 12.1}
	\includegraphics[scale=0.8]{ch12zdj12}
\end{wrapfigure}  
}
\end{parcolumns}
\footnotetext{\textsuperscript{1}R. Baierlein, Newtonian Dynamics (McGraw Hill, New York, 1983), p. 88$-$93.}
\newpage
\begin{minipage}[t]{10cm}
\section{Limit Cycles and Attractors}
The limit cycle you observed in this problem is a simple example of an attractor
in phase space. An attractor is a curve in the phase$-$space of the differential
equation to which many different solutions (having different initial conditions)
tend. For instance, for the damped un$-$driven harmonic oscillator the attractor is
just the state of no motion: $x = 0, v = 0$, because all solutions end up here. For the
driven damped harmonic oscillator the attractor is more interesting: it is the final
driven steady state of the oscillator, which looks like an ellipse in phase space.
Since this attractor is not a single point, we also call it a limit$-$cycle. For the van
Der Pol equation the attractor is the oddly$-$shaped curve (or limit$-$cycle) in the
$(x, v)$ phase space to which all solutions tend.\\
Sometimes an attractor is not a single curve, but rather a very complex structure, like the famous Lorenz attractor (which you can explore a little bit by typing
lorenz at the command prompt in Matlab). These kind of attractors are called
strange attractors, and are examples of chaotic systems. We\textquotesingle ll study chaos later in
this lab and you will see other examples of attractors, but none of the attractors
encountered in this lab are strange attractors (except the Lorenz attractor).
\subsection*{\small{P12.2}}\label{P12.2}
 Now let\textquotesingle s add a driving force to the van der Pol oscillator, like this:
\begin{equation}\label{12.2}
\ddot{x}-\epsilon(\ell ^{2} -x^{2})\dot{x}+{\omega _0}^{2}x=Acos\omega t
\end{equation}
Using $\ell = 1, \epsilon = 2, \omega _0 = 1.3, and \omega = 1.4$, gradually increase A from 0 to 1.5
and watch what happens to the power spectrum of x(t). Change A by steps
large enough to see qualitative changes, i.e., don\textquotesingle t do A = 0.01, A = 0.02,
A = 0.03, etc.
You should find that as A is increased the limit cycle becomes fuzzy and that
the power spectrum becomes increasingly filled with spikes. Finally, around
A = 1.25 $\rightarrow$ 1.27 the power spectrum becomes so complicated that it is fuzzy
too (use the zoom feature on the spectrum to see that the spectrum is made
up of many tiny peaks). And then, quite abruptly, at about A = 1.28 the
oscillator becomes slaved to the drive, meaning that the oscillator vibrates
at the driving frequency $\omega  = 1.4$ and its harmonics, making the spectrum
simple again. (Look carefully at the power spectrum to see that this is true).
\section{Entrainment}
The kind of behavior illustrated in \hyperref[P12.2]{P12.2} is called entrainment, in which an oscillator becomes synchronized to another periodic signal. An important example of
a system like this is the human heart. The heart has an external source of power,
has an unstable resting state (it wants to beat rather than sit still), and, normally, a
stable limit cycle (thump$-$Thump, thump$-$Thump,...). Sometimes this stable limit
cycle becomes irregular, in which case it is desirable to supply a periodic driving signal via a pacemaker which, if strong enough, can force the heart to become
entrained with it, restoring a stable limit cycle, albeit at a frequency determined
by the pacemaker rather than by the physical needs of the patient.
\end{minipage}
\newpage
\begin{parcolumns}[colwidths={2=0.3\linewidth}]{2}
\colchunk{
\section{Dynamical Chaos}
Now let\textquotesingle s switch gears a bit and take a brief tour through one of the most exciting
areas in the study of differential equations: dynamical chaos. A chaotic system is
one where dynamical variables (e.g. position and velocity) behave in seemingly
erratic ways and exhibit extreme sensitivity to initial conditions. Chaotic systems
are deterministic, since a given set of parameters and initial conditions reproduce the same motion, but it is usually difficult to predict how tiny variations in
parameters or initial conditions will affect the motion.\\
Chaotic systems have been known and studied for a long time. For instance,
it comes as no surprise that when you have 1023 atoms bouncing around inside a
container, hitting the walls and hitting each other, that the motion of any given
atom is pretty chaotic. But in the middle of the twentieth century it was discovered
that even simple systems can be chaotic. For instance, here is the apparently
nice, smooth, and well$-$behaved differential equation for the driven damped
pendulum:
\begin{equation}\label{12.3}
\dfrac{d^{2}\theta}{dt^{2}}+\gamma \dot{\theta}+{\omega _0}^{2}sin\theta =Acos\omega t
\end{equation}
This system has only two degrees of freedom (way less than $10^{23}$) and all of the
functions that appear in it are nice and smooth. But for certain choices of A, $\omega$,
$\omega _0$, and $\gamma $ the solutions of this differential equation are almost as unpredictable
as the motion of an atom in a gas.\\
Chaos is hard to study because that old standby of physical theory, the formula, is not of much help. If we had a formula for the solution of this differential
equation its behavior would be perfectly predictable and un$-$chaotic. Since the
dynamics in chaotic systems are not represented by analytic formulas, their solution had to wait for computers to be invented and to become powerful. The
computers we will use in this laboratory are more powerful than the computers
we used to send men to the moon and to design nuclear weapons in the 1960s
and 1970s, so we have all the computing power we need to at least be introduced
to this fascinating field.
\subsection*{\small{P12.3}}\label{P12.3}
 A simple system in which chaos can be observed is a particle moving in a
potential well with two low spots:
\begin{equation}\label{12.4}
U(x)=-\dfrac{x^{2}}{2}+\dfrac{x^{4}}{4}.
\end{equation}
\begin{enumerate}[label=\alph*]
\item  Plot this potential vs. x and locate the two stable equilibrium points
(the one in the middle is unstable).
\item Let a particle have mass m = 1 and use the force relation
\begin{equation}\label{12.5}
f_x=-\dfrac{\partial U}{\partial x}
\end{equation}
to derive the equation of motion of the particle. Then write a Matlab
script and a function that employs ode45 to solve for the motion of the
particle. Use options=odeset(\textquotesingle RelTol\textquotesingle ,1e$-$6) to set the accuracy
of ode45. Try several different initial conditions and watch how the
particle behaves in this double well. Look at the motion in phase space
for enough different initial conditions that you can see the transition
from motion in one well or the other to motion that travels back and
forth between the wells.
\item Now add a driving force of the form $F = Acos 2t$ and also include a
linear damping force $Fd_{damp} = −m\gamma \dot{x}$ with $\gamma  = 0.4$. Use initial conditions $x(0) = 1, v(0) = 0$, and make a series of runs with A gradually
increasing until you observe chaotic behavior. (The transition from
regular motion to chaos occurs between A = 0.7 and A = 0.8). Run
from t = 0 to t = 1000. A plot of x(t) should show random jumping
between the left and right sides of the double well, as illustrated in
\hyperref[Figure 12.2]{Figure 12.2}. For each run make a plot of the power spectrum of x(t).
Show the TA how your plots illustrate intermittency and 1/f noise
(described below).
\item  With A = 0.9 do two runs, one with initial conditions x(0) = 1, v(0) = 0,
and the other with x(0) = 1.000001 and v(0) = 0. Plot x(t) for each
of these cases, and explain to the TA how these plots illustrate the
butterfly effect (described below).
\end{enumerate}
}
\end{parcolumns}
\begin{parcolumns}[colwidths={2=0.3\linewidth}]{2}
\colchunk
{
	\section{Intermittency, 1/f Noise, and the Butterfly Effect}
	The random switching back and forth between equilibrium positions observed in
\hyperref[P12.3]{P12.3}(c) is called intermittency and is one of standard ways that regular systems
become chaotic. As the motion becomes chaotic you should also see an increase
in the spectrum near $\omega = 0$. This low frequency peak in the spectrum is one of the
symptoms of chaos (called \textquotesingle\textquotesingle 1/f \textquotesingle\textquotesingle ) and is a direct consequence of the slow
random switching of intermittency.\\
Another hallmark of chaotic systems is the so-called \textquotesingle\textquotesingle butterfly effect\textquotesingle\textquotesingle (illustrated in \hyperref[P12.3]{P12.3}(d)), where very small changes in the initial conditions cause
large differences in the motion. This effect was discovered by Edward Lorenz
(for whom the Lorenz attractor is named), who was a meteorologist that studied
numerical models for weather prediction in the early 1960s. He noticed that very
tiny differences in initial conditions (too small to even be measured) led to vastly
different outcomes in his model. The effect gets its name from a talk that he gave
in 1972 titled \textquotesingle\textquotesingle Predictability: Does the Flap of a Butterfly\textquotesingle s Wings in Brazil set off a
Tornado in Texas?\textquotesingle\textquotesingle .
\subsection*{\small{P12.4}}\label{P12.4}
\begin{enumerate}[label=\alph*]
\item  Make a phase space plot for the system in \hyperref[P12.3]{P12.3}c) with A = 0.96. You
should find that the chaotic behavior quiets down and is replaced by
a limit cycle in phase space. It will be difficult to see the limit cycle on
the phase space plot because of the messy transients at the beginning.
To eliminate the transients make the phase space plot like this (We chose to skip the first 60 percent $-$ you can try your own value):
\begin{lstlisting}[language=Matlab][numbers=none]
N=length(x);
n1=ceil(.6*N); % n1 starts 60% into the array
plot(x(n1:N),v(n1:N));
\end{lstlisting}
\item Make another phase space plot at A = 1.30. The single limit cycle
should be replaced by a 2$-$cycle (two loops in phase space before
repeating);\\
Note: to really see the multiple character of these cycles, use the zoom
feature in the plot window to look carefully at them, especially near
the tight loops. This is shown in \hyperref[Figure 12.3]{Figure 12.3} for the 4$-$cycle state. In
the upper window the full time history is shown from the beginning
while in the lower window the late$-$time final state in the window
from the upper frame is shown. There are clearly 4 repeated loops
in phase space, so this is called a 4$-$cycle. This is an example of the
famous \textquotesingle\textquotesingle period$-$doubling route\textquotesingle\textquotesingle to chaos, as well as an example of
regular behavior in a region of parameter space where you might have
expected chaos.
\item Make phase space plots of the limit cycle at A = 1.36 (a 4$-$cycle state),
which is then replaced by an 8$-$cycle at A = 1.371, and then chaos
takes over again.\\
If you run with A = 1.97, 1.99, and 2.0, you will see chaos disappear to
be replaced by a 2$-$cycle, a 4$-$cycle, and an 8$-$cycle. Beyond 2 there is
chaos again.\\
At A = 3 the amplitude is large enough that the oscillator becomes
slaved to the drive and we have entrainment. You might think that
large A would always cause entrainment, but A = 50 is chaotic, and
there are probably lots of 2,4,8,... cycles and chaotic regions as A is
varied. We ran out of patience; let us know what you find.\\
Note: when you run with A = 1.36 your phase$-$space picture may look
like an upside$-$down left$-$right flipped version of \hyperref[Figure 12.3]{Figure 12.3}. This is OK$–$
the differential equation is almost unchanged if (x, v) is replaced with
$(−x,−v)$. The only difference is that the driving term is replaced by
its negative, which is equivalent to a phase shift of $\pi$. Such a phase
shift could occur by having the oscillator start up in a different way,
which might easily happen if your initial conditions were not exactly
the same as ours. This flipped$-$over state is to be expected on physical
grounds. Our picture has tight loops on the left and big loops on the
right, but the potential is left$-$right symmetric; there should be another
state with tight loops on the right and big ones on the right as well.
\end{enumerate}
}
\colchunk{
\begin{wrapfigure}{r}{4cm}
\captionsetup{singlelinecheck=off, justification=raggedright, format=hang}
\caption{Intermittent random
bouncing between the two wells.}
\label{Figure 12.2}
\includegraphics[scale=0.8]{ch12zdj2}
\end{wrapfigure}
\begin{wrapfigure}{r}{4cm}
\caption{Full time history, then
final 4$-$cycle state from the small
window. There are fewer loops in
the lower trace because it is the
final state; the extra loops in the
box in the upper trace are from
early times.}
\label{Figure 12.3}
\includegraphics[scale=0.6]{ch12123}
\end{wrapfigure} 
}
\end{parcolumns}
\begin{parcolumns}[colwidths={2=0.3\linewidth}]{2}
\colchunk
{
\section{Fractals}
\subsection*{\small{P12.5}}\label{P12.5}
\begin{enumerate}[label=\alph*]
\item The Fibonacci sequence $F_n$ is defined by
\begin{equation}\label{12.6}
	F_1 =1~~;~~F_2=1~~;~~F_n=F_{n-1}+F_{n-2} ~for ~n\geq 3
\end{equation}
(The first few numbers in the sequence are 1,1,2,3,5,8,13,...). Write
a loop that fills the array Fn with the first 100 values of the Fibonacci
sequence.
\item Now define an array x that goes from $−2\pi$ to $2\pi$ with 50,001 equally
spaced values, like this:
\begin{lstlisting}[language=Matlab][numbers=none]
h=4*pi/50000;
x=-2*pi:h:2*pi;
\end{lstlisting}
Then write a loop that evaluates the Fourier$-$like series
\begin{equation}\label{12.7}
G(x)=\sum_{n=1}^{100}\dfrac{cos(F_n x)}{F_n}.
\end{equation}
Plot this function vs. x and carefully observe its shape (this function
is shown in \hyperref[Figure 12.4]{Figure 12.4}). Then use the zoom feature to more closely
examine some of the smaller mountain peaks to discover that each
mountain peak contains smaller versions of itself.\\
If you zoom in too much you will run out of points, so now plot the
function again using 50,001 points between x = 3.1 and x = 3.2, and
zoom in again. This kind of curve is called a fractal, or fractal curve\textsuperscript{2}
and such curves are important in chaos theory.
\end{enumerate}
}
\colchunk
{
\raggedright
\begin{wrapfigure}{r}{4cm}
\caption{The function plotted
in \hyperref[P12.5]{P12.5}}
\label {Figure 12.4}
\includegraphics[scale=0.8]{ch12ks}
\end{wrapfigure} 
}
\footnotetext{\textsuperscript{2}S. N. Rasband, Chaotic Dynamics of Nonlinear Systems (John Wiley and Sons, New York, 1990),
Chap. 4, and http://sprott.physics.wisc.edu/fractals.htm}
\end{parcolumns}
\end{document}